%% $Id: paper.tex,v 1.9 2005/05/30 18:19:18 jbednar Exp $

\documentclass[10pt]{article}   % LaTeX 2e document

%\PassOptionsToPackage{draft}{hyperref} %% Turn off hyperref links

%%\usepackage{nnonecolumn}      % Research paper defaults, single column
\usepackage{nntwocolumn}        % Research paper defaults, double column
\usepackage{times}              % Selects the Times Roman font
\usepackage{fancyhdr}           % Support for to-appear macros
\usepackage[sort&compress]{natbib}             % Bibliography and citation package
%%\usepackage{hypernat}         % May be needed to get natbib working with hyperref
%\usepackage{picins}             % Optional; for figures with text flowing around a graphic

%% Use the nnonecolumn package above to make document single-column 
%% (e.g. for journal submissions).  Many journals and conferences will
%% supply their own style files, which would then be used in place of
%% nnonecolumn and nntwocolumn above.  
%%
%%   Two column figures:
%% This document contains only single-column figures, but sometimes
%% figures need to span both columns even in two-column mode.  If so
%% just change those figures to use \begin{figure*}...\end{figure*}
%% instead of \begin{figure}...\end{figure}.


%%%%%%%%%%%%%%%%%%%%%%%%%%%%%%%%%%%%%%%%%%%%%%%%%%%%%%%%%%%%%%%%%%%%%%%%%%%%%%%
%%%%%%%%%%%%%%%%%%%%%%%%%%%%%%%%%%%%%%%%%%%%%%%%%%%%%%%%%%%%%%%%%%%%%%%%%%%%%%%
% Changebar support (optional)
% 
% Use changebar.sty v3.3i or later; earlier ones (shipped with at
% least Red Hat Linux 6.0 and Debian Linux 2.0) are very buggy.
%
% Marks changes relative to previous revision(s), for the convenience of
% reviewers or collaborators such as a supervising professor. Changes
% can be marked in the source by \cbstart and \cbend macros, which can
% be generated automatically by chbar.sh, or by enclosing them in a
% begin{changebar} end{changebar} environment.  
% 
% The environment form is more reliable, since LaTeX ensures that it
% does not span any parenthesis level or environment boundaries (by
% complaining when that assumption is violated).  
% 
% If using the \cb macros, even if generated by chbar.sh, be sure that
% if a \cbstart is inside a "{" then the matching \cbend is also
% inside the "}". Otherwise some incorrect areas will be marked,
% e.g. an entire figure will be marked when only a small portion of
% the caption changed.  If you still have problems, e.g. figures
% incorrectly marked, one thing that often helps is to go through
% and ensure that each figure to be marked is surrounded immediately 
% with a \cbstart/\cbend pair, even if the entire section is to 
% be marked change. (In the latter case, put a \cbend before the
% figure, and a \cbstart after, so that the text and the figure
% will be covered by different start/end pairs.)
%
%\usepackage[dvips,leftbars]{changebar}
%%\nochangebars % Disable bars


%%%%%%%%%%%%%%%%%%%%%%%%%%%%%%%%%%%%%%%%%%%%%%%%%%%%%%%%%%%%%%%%%%%%%%%%%%%%%%%
%%%%%%%%%%%%%%%%%%%%%%%%%%%%%%%%%%%%%%%%%%%%%%%%%%%%%%%%%%%%%%%%%%%%%%%%%%%%%%%
%% HTML support (optional)
\usepackage{html}               % Allows explicit links, conditional text, etc.
\usepackage{heqn}               % Makes equations work better in HTML

\begin{htmlonly}
  %% Don't separate subsections, but do separate sections and chapters
  \HTMLset{MAX_SPLIT_DEPTH}{4}
  
  %% Uncomment this line instead for very short documents which should all 
  %% be put on one page in the HTML version
  %%\html{\HTMLset{MAX_SPLIT_DEPTH}{0}\HTMLset{NO_NAVIGATION}{1}}
  
  %% Determine how large the figures should be
  \HTMLset{FIGURE_SCALE_FACTOR}{3.0}
  
  %% Make "Up" button at top level go to the site which describes this file
  \HTMLset{EXTERNAL_UP_TITLE}{TAE Research}
  \HTMLset{EXTERNAL_UP_LINK}
  {http://www.cs.utexas.edu/users/jbednar/tae.html}

  \HTMLset{EXTERNAL_DOWN_TITLE}{James A. Bednar's Research}
  \HTMLset{EXTERNAL_DOWN_LINK}
  {http://www.cs.utexas.edu/users/jbednar/research.html}
\end{htmlonly}

%%%%%%%%%%%%%%%%%%%%%%%%%%%%%%%%%%%%%%%%%%%%%%%%%%%%%%%%%%%%%%%%%%%%%%%%%%%%%%%
%%%%%%%%%%%%%%%%%%%%%%%%%%%%%%%%%%%%%%%%%%%%%%%%%%%%%%%%%%%%%%%%%%%%%%%%%%%%%%%

%% You can override the style definitions here if you need to squeeze
%% more text in within page limits, or you prefer wider margins, etc.
%\setlength{\textheight}{9in}
%\setlength{\topmargin}{0in}
%\setlength{\textwidth}{6.5in}
%\setlength{\oddsidemargin}{0.0in}
%\setlength{\evensidemargin}{0.0in} % only used in twoside style

% Input locally-defined macros, e.g. for picture inclusion
%% $Id: nnmacros.tex,v 1.28 2005/09/28 21:18:37 jbednar Exp $
%%
%% Standard set of LaTeX macros and declarations for the NN group
%%
%% Originally based on risto's com.tex
%%
%% Because of \usepackage{hyperref}, this file should be included
%% *after* all other packages.


%%%%%%%%%%%%%%%%%%%%%%%%%%%%%%%%%%%%%%%%%%%%%%%%%%%%%%%%%%%%%%%%%%%%%%%%%%%%%%%
%% Add ability to check for PDFTeX

\newif\ifpdf
\ifx\pdfoutput\undefined
  \pdffalse
\else
  \pdfoutput=1
  \pdftrue
\fi

%% Macro useful for figuring out if a TeX command has been defined
%%
\def\ifundefined#1{\expandafter\ifx\csname#1\endcsname\relax}

%%%%%%%%%%%%%%%%%%%%%%%%%%%%%%%%%%%%%%%%%%%%%%%%%%%%%%%%%%%%%%%%%%%%%%%%%%%%%%%
%% Packages required


% Because of this call, the "graphics" interface can't be used in
% conjunction with this file.  
%
\ifpdf\PassOptionsToPackage{pdftex}{graphicx}\fi
\usepackage{graphicx}

%% If you want to supply options to the hyperref package, you can add :
%% 
%% \PassOptionsToPackage{...}{hyperref} 
%%
%% in your source file before any \usepackage commands.  To turn
%% off hyperlinks altogether, you can use: 
%%
%% \PassOptionsToPackage{draft}{hyperref} %% Turn off hyperref links
%%
\ifpdf
%% Highlights links in PDF files
\usepackage{hyperref}
%% Highlights bibliography links
\newcommand{\biblink}[2]{\htmladdnormallink{#1}{#2}}
\fi

% From the LaTeX2HTML distribution.
%\usepackage{html}%   
%
% If you don't have or want LaTeX2HTML when running this at home, just
% copy /u/nn/tex/latex2html/texinputs/html.sty to somewhere   
% in your TEXINPUTS path and ignore the rest of LaTeX2HTML.


%%%%%%%%%%%%%%%%%%%%%%%%%%%%%%%%%%%%%%%%%%%%%%%%%%%%%%%%%%%%%%%%%%%%%%%%%%%%%%%
%% From risto's com.tex:

%%% Generates a \tt quotation without indentation
%\def\quotation{\list{}{\listparindent 0.0em
%    \setlength{\parindent}{0.0in}
%    \renewcommand{\baselinestretch}{1}\small\normalsize\tt
%    \renewcommand{\arraystretch}{1}
%%    \itemindent\listparindent
%    \setlength{\leftmargin}{0.0in}
%    \parsep 0pt plus 1pt}\item[]}
%\let\endquotation=\endlist

%% I like to tighten up my captions and tables
\newcommand{\mycapdefs}{\renewcommand{\baselinestretch}{1}\small}
\newcommand{\mytabdefs}{\renewcommand{\arraystretch}{1}\small}

%% Get it?
\hyphenation{Miik-ku-lai-nen}

%% I don't like pages with only figures on them
\renewcommand{\topfraction}{0.9}
\renewcommand{\floatpagefraction}{0.8}
\renewcommand{\dbltopfraction}{0.9}
\renewcommand{\dblfloatpagefraction}{0.8}
\renewcommand{\textfraction}{.15}

%% Simple PostScript picture inclusion macros using the graphicx package
%%
%% These three commands have been provided for backwards compatibility
%% only.  You should not use them in new documents, because the 
%% non-standard argument format causes trouble with packages that expect
%% LaTeX-style  arguments, such as AucTeX, emacs font-lock, and
%% LaTeX2HTML.  Instead use the \pspic or \pdfpic macros (below).
%%
\begin{latexonly}
\def\putpicture   #1 (#2)\centerline{\includegraphics[totalheight=#2]ps/#1.ps}
\def\hputpicture  #1 (#2){\centerline{\includegraphics[width=#2]{ps/#1.ps}}}
\def\hvputpicture #1 (#2 by #3){\centerline{\includegraphics[width=#2,totalheight=#3]{ps/#1.ps}}}
\end{latexonly}


%%%%%%%%%%%%%%%%%%%%%%%%%%%%%%%%%%%%%%%%%%%%%%%%%%%%%%%%%%%%%%%%%%%%%%%%%%%%%%%
%% \scribbledraft: Provides a single-column mode with lots of space
%% to write comments on, e.g. for giving to a coauthor.
%%
%% To use, add \scribbledraft before \begin{document} but after
%% all your \usepackage commands.
%%
%% Tips:
%%
%% If the formatting doesn't look right, you may have to comment out
%% your document style (e.g. \usepackage{<confname>}) and instead use
%% \usepackage{nntwocolumn}.  Usually it will only be the first 
%% page that has problems, though; that's where style files usually
%% mess with the page widths and lengths.
%%
%% If you get lines sticking past the edge of the column, try adding
%% \sloppy just after \scribbledraft.
%%
%% If your document has figure or figure* environments that come out much
%% too small in \scribbledraft mode, you may be able to fix them by
%% changing them to fullfigure or fullfigure* environments, which are
%% defined below.  However, you may find that the result is no longer
%% centered properly or is the wrong width.  If so, you might need to add
%% a \setlength\originalwidth{6.5in} or \setlength\marginoffset{0in}
%% after \scribbledraft, adjusting those two values to fix the width and
%% left margin.  
%%
\newcommand{\scribbledraft}{
  \newlength\marginoffset\setlength\marginoffset{-0.5in}
  \newlength\nominalwidth\setlength\nominalwidth{3.5in}
  \newlength\nominalmargin\setlength\nominalmargin{0.5in}
  \setlength\topmargin{0.5in}
%% \setlength\oddsidemargin{0.5in}
%% \setlength\evensidemargin{0.5in}
\setlength\oddsidemargin{\nominalmargin}
\setlength\evensidemargin{\nominalmargin}
\setlength\textheight{8in}
\setlength\textwidth{\nominalwidth}
\setlength\footskip{20pt}
\setlength\marginparwidth{2.5in}
\pagestyle{plain}
\onecolumn
% Some styles (e.g. cogsci, nntwocolumn) call \twocolumn in \maketitle,
% so here we define it as a no-op
\renewcommand{\twocolumn}[1][]{##1\bigskip\bigskip} 
%%
\renewcommand{\forcedoublecolumnfigurestart}
{\setlength\textwidth{\originalwidth}%
\hspace*{\marginoffset}%
\begin{minipage}{\originalwidth}}
%%
\renewcommand{\forcedoublecolumnfigureend}
{\end{minipage}\setlength\textwidth{\nominalwidth}}
}
%% Save original value here for later
\newlength{\originalwidth}\setlength{\originalwidth}{\textwidth}
%%
\newcommand{\forcedoublecolumnfigurestart}{}
\newcommand{\forcedoublecolumnfigureend}{}
%%
\newenvironment{fullfigure}[1][tbp]
{\begin{figure}[#1]\forcedoublecolumnfigurestart}
{\forcedoublecolumnfigureend\end{figure}}
%%
\newenvironment{fullfigure*}[1][tbp]
{\begin{figure*}[#1]\forcedoublecolumnfigurestart}
{\forcedoublecolumnfigureend\end{figure*}}



%%%%%%%%%%%%%%%%%%%%%%%%%%%%%%%%%%%%%%%%%%%%%%%%%%%%%%%%%%%%%%%%%%%%%%%%%%%%%%%
%% Conditional picture inclusion macros using the graphicx package
%%
%% If you use \pdfpic (below) when specifying a picture, you
%% can use \psdraftlevel (also below) to specify which ones to include
%% for a printout (e.g. to save print time or file size, or to print
%% color pages separately).  It's also handy to have the type of picture
%% declared in the source file this way, as a clue to what type of
%% picture is involved.  Just ignore this if it is not needed.

%% First, declare some symbolic constants representing the different
%% classes of images.  These definitions shouldn't ever need changing, 
%% but new ones can be added in between if needed.
%%  
\def\psnone{100}        % No picture should be included
\def\psline{80}         % Black and white line drawings only
\def\psmono{60}         % Monochrome images only
\def\pscolororbw{40}    % Images for which color is optional
\def\pscolor{20}        % Images for which color is required
\def\psall{0}           % All pictures should be included


%% The \psdraftlevel command determines which pictures should actually
%% be included, while other pictures merely have their bounding
%% box drawn in to save time or file size, or to print on printers
%% with insufficient memory.  It is optional, but if used, should
%% presumably be in the preamble. Each of the above levels is
%% inclusive of the preceding ones (those with lower numbers), so
%% \mono includes line drawings too, and \color includes line, mono,
%% and color pictures. 
%%
%% Example:  To include all pictures (the default),
%%   \psdrafttype{\psall}
%%
%% Example:  To include all pictures which make sense in monochrome,
%%   \psdrafttype{\pscolororbw}
%%
%% Example:  To include only line drawings, i.e. pictures likely to be
%% very small in file size, 
%%   \psdrafttype{\psline}
%%
%% Below, 0 is hardcoded instead of \psall because of problems 
%% with some versions of latex2html
\newcount\psdraftlevel \psdraftlevel=0
\newcommand{\psdrafttype}[1]{\psdraftlevel=#1}


%% Including PDF, JPG, PNG, or TIF pictures in PDFLaTeX documents,
%% (as well as EPS pictures in LaTeX documents)
%%
%% \pdfpic[category]{width}{height}{[dir/]filename}{filetype}
%%
%%
%% Arguments:
%%
%% The "category" argument is optional; if present it should indicate the
%% type of image the file represents, i.e. \psline for a line drawing,
%% \pscolor for a color photo, \pscolororbw for an image that is readable
%% either in color or monochrome, \psnone for a blank space, etc.  For 
%% a given run, the actual image is included only if the the
%% \psdrafttype is low enough.  For example, if \psdrafttype}{\psmono}
%% has been specified, all line and monochrome pictures are included,
%% but color pictures will be represented only by a bounding box.  If
%% "category" is omitted then the image will always be displayed,
%% regardless of draftlevel.
%%
%% Either or both of the length arguments "length" and "width" can be 
%% omitted if they are replaced by "!", which indicates that that
%% dimension should scale proportionally.  (If both are "!", then the
%% image is left at the natural size of the image, whatever that might
%% be for a given filetype.)  A dummy image (type \psnone) must have
%% both "length" and "width" specified, since they cannot be read from
%% a file.
%%
%% The argument "[dir/]filename" should be the complete path to your
%% picture file, minus the final extension, which in turn goes in the 
%% {filetype} argument.  For instance, to access "./images/pretty.jpg",
%% use \pdfpic{!}{!}{images/pretty}{jpg}.  
%%
%% 
%% Tips and examples:
%% 
%% Filetypes (i.e., extensions) of jpg, png, pdf and tif are currently
%% supported. Of these you should usually use pdf for line drawings,
%% jpg for photographs, and tif or png for non-photograph bitmap images.
%% 
%% Usually, it is best to specify that a picture be fixed in size
%% horizontally if it extends across the page, so that it still fits
%% on the page properly if the margins, number of columns, etc. are
%% changed.  
%%
%% Example: To insert a monochrome PNG picture images/sample.png, scaled
%% to be as wide as the current page,        
%%   \pdfpic[\psmono]{\textwidth}{!}{images/sample}{png}
%%
%% Example: To insert a line drawing images/sample.pdf, scaled to be as
%% half as wide as the page, rotated 90 degrees counterclockwise, and
%% framed with a thin, snug outline,
%%   \frame{\rotatebox{90}{\pdfpic[\psline]{!}{0.5\textwidth}{images/sample.pdf}{pdf}}}
%%
%% Example: To insert a dummy 3" x 2" space for pasting in a picture 
%% later, when no image file actually exists on disk,
%%   \pdfpic[\psnone]{3in}{2in}{Paste feller:science96 figure 3 here}{}
%%
%% If this command is used under LaTeX, the filetype is ignored and 
%% one of type eps and ending in .ps is assumed.  If you do use such EPS 
%% files, make sure that they are truly EPS (i.e., have a bounding box)
%% and not just PostScript.
%%
%%
\newcommand{\pdfpic}[5][\psall]{\includepdfpic[#1]{#2}{#3}{#4}{#5}{}}%


%% Same as \pdfpic but clips to boundary of box
\newcommand{\pdfpicclip}[5][\psall]{\includepdfpic[#1]{#2}{#3}{#4}{#5}{clip=True,}}%



%% Old-style interface for inserting PostScript pictures into LaTeX
%% documents:
%%
%%
%% This interface only supports PDF and EPS filetypes, and is included
%% for backwards compatibility with older documents developed in LaTeX.
%% Such documents can be converted to work with PDFLaTeX simply by
%% converting all of the EPS files they include to PDF.  For instance,
%% (in csh):
%%
%% foreach psfile (ps/*.ps) 
%%   epstopdf "${psfile}"
%% end
%%
%% That way PDFLaTeX will find the ps/*.pdf files, and LaTeX will find
%% the ps/*.ps files.  Be sure to check the generated .pdf figures
%% carefully; sometimes the translation process generates visible
%% artifacts.  For instance, sometimes you can see fringes around
%% sharp, high-contrast borders in images, like those when converting
%% bitmaps to JPEG.
%%
%% 
%% This macro is used just like \pdfpic, except that you should omit the
%% filetype (because only one filetype is supported for each platform)
%% and the directory (which is assumed to be ./ps/):
%%
%% \pspic[category]{width}{height}{filename}
%%
\newcommand{\pspic}[4][\psall]{\pdfpic[#1]{#2}{#3}{ps/#4}{pdf}}
\newcommand{\pspicclip}[4][\psall]{\pdfpicclip[#1]{#2}{#3}{ps/#4}{pdf}}




%%%%%%%%%%%%%%%%%%%%%%%%%%%%%%%%%%%%%%%%%%%%%%%%%%%%%%%%%%%%%%%%%%%%%%%%%%%%%%%
%%
%% Implementation for \pdfpic, etc.; shouldn't usually 
%% be called directly.  Arguments are same as for \pdfpic
%% except that a 6th argument is accepted, and passed
%% directly to \includegraphics
%%
\newcommand{\includepdfpic}[6][\psall]{%
\def\pdfpicFileType{#5}%
\def\pdfpicFileExt{#5}%
%%
%% Assume .ps under LaTeX
\ifpdf\else% 
\def\pdfpicFileType{eps}%
\def\pdfpicFileExt{ps}%
\fi%
%%
%% Dummy picture: empty box (not even a frame).
%% Unspecified heights default to \unitlength
\ifx#1\psnone%
\resizebox{#2}{#3}{\begin{picture}(1,1)\end{picture}}%
%%
%% Draft picture: framed box with filename inside.
%% This looks fine with dvips, but it confuses xdvi
%% for some reason, which draws an extra box with the
%% natural size of the image. 
\else{\ifnum#1<\psdraftlevel%
\frame{\resizebox{#2}{#3}{\includegraphics[#6draft=true,type=\pdfpicFileType,ext=.\pdfpicFileExt,read=*,command=##1]{#4}}}%
\else%
%%
%% Normal picture: image in the specified format.
\resizebox{#2}{#3}{\includegraphics[#6type=\pdfpicFileType,ext=.\pdfpicFileExt,read=*,command=##1]{#4}}\fi}\fi%
}

%% Simpler version to keep LaTeX2HTML from getting confused
\begin{imagesonly}
\renewcommand{\pspic}[4][\psall]{\resizebox{#2}{#3}{\includegraphics{ps/#4.ps}}}
\end{imagesonly}



%%%%%%%%%%%%%%%%%%%%%%%%%%%%%%%%%%%%%%%%%%%%%%%%%%%%%%%%%%%%%%%%%%%%%%%%%%%%%%%
%%>> To do (6/2002): 
%% It would be nice to have some sort of facility for automatically 
%% handling conditional text for different document types.  Here's
%% a sketch, but it is not legal LaTeX code:
%%
%% \newcommand{\definedocumenttype}[2]{
%%   \expandafter\def\csname dt#1\endcsname{#2}%
%%   \newcommand{not#1}[1]{}%
%%   \newcommand{#1only}[1]{}%
%% }
%% \definedocumenttype{book}{100}          % Book
%% \definedocumenttype{thesis}{80}         % PhD. or Master's thesis
%% \definedocumenttype{techreport}{60}     % Technical report
%% \definedocumenttype{preprint}{45}       % Journal preprint (to be passed around, not submitted)
%% \definedocumenttype{journal}{40}        % Journal paper (as published, or to be submitted)
%% \definedocumenttype{conf}{20}           % Conference paper
%% \definedocumenttype{article}{0}         % Default; generic article
%% 
%% \newcount\documenttype \documenttype=0
%% \newcommand{\setdocumenttype}[1]{%
%%   \documenttype=#1%
%%   \newcommand{not#1}[1]{#1}%
%%   \newcommand{#1only}[1]{}%
%% }
%%
%% Usage:
%%  \setdocumenttype{\book}  % Or any other
%%  \bookonly{Text for the book version}
%%  \notbook{Text for all versions except the book}
%%
%% All 'only' and 'not' comparisons should be available
%% even if only a single document type is available.  It should
%% be legal to define multiple document types.  However, it's not
%% clear how to implement any of this.


%%%%%%%%%%%%%%%%%%%%%%%%%%%%%%%%%%%%%%%%%%%%%%%%%%%%%%%%%%%%%%%%%%%%%%%%%%%%%%%
%%  Standard lengths, used to keep documents uniform throughout
 
%% Set size of windowpane to fit one, two, etc. square panes per line  
%%
\newlength{\oneplotwidth}           \setlength{\oneplotwidth}         {\textwidth}
\newlength{\twoplotwidth}           \setlength{\twoplotwidth}         {0.46\textwidth}
\newlength{\twoplotheight}          \setlength{\twoplotheight}        {\twoplotwidth}
\newlength{\threeplotwidth}         \setlength{\threeplotwidth}       {0.31\textwidth}
\newlength{\threeplotheight}        \setlength{\threeplotheight}      {\threeplotwidth}
\newlength{\fourplotwidth}          \setlength{\fourplotwidth}        {0.22\textwidth}
\newlength{\fourplotheight}         \setlength{\fourplotheight}       {\fourplotwidth}
\newlength{\fiveplotwidth}          \setlength{\fiveplotwidth}        {0.18\textwidth}
\newlength{\fiveplotheight}         \setlength{\fiveplotheight}       {\fiveplotwidth}
 
%% Set size of windowpane to fit the specified configuration of square
%% panes per page  
%%
\newlength{\twobytwoplotwidth}      \setlength{\twobytwoplotwidth}    {0.46\textwidth}
\newlength{\threebyfourplotwidth}   \setlength{\threebyfourplotwidth} {0.25\textwidth}
\newlength{\fourbythreeplotwidth}   \setlength{\fourbythreeplotwidth} {0.171875\textheight}
%% Note that the portrait page above with 12 square panes is named as 
%% a "width", but it is actually limited in the vertical direction
%% for typical US paper proportions, assuming there is a 1-inch caption.
 
 
%% Plot width that xmgr graphs should use
%%
%% This value is appropriate for output that has been run through
%% ps2epsi to fix the bounding boxes.  Xmgr computes them too
%% generously, so this would need to be some smaller value otherwise.
%%
\newlength{\xmgrplotwidth}          \setlength{\xmgrplotwidth}{\textwidth}
 
 
%% Blank variable available for various purposes
%%   Must use \setlength or \settowidth to give it 
%%   a value before using it
%%   E.g.: \settowidth{\templength}{Some text}
%%
\newlength{\templength}

 

%%%%%%%%%%%%%%%%%%%%%%%%%%%%%%%%%%%%%%%%%%%%%%%%%%%%%%%%%%%%%%%%%%%%%%%%%%%%%%%
%% Support for captioned subfigures
%%
%% To enable these commands, add "\usepackage[tight,center]{subfigure}"
%% before loading this file.  Then define subfigures with \sfobject, \sfpic,
%% or \sfpicframe, and refer to them with \sfref or \sfrefnp.
%%
%% Requires subfigure package version 2.1.2 or higher.  See the subfigure 
%% documentation for more details, such as how to change the caption 
%% separation, fonts, etc.:
%% http://www-2.cs.cmu.edu/afs/cs/usr/sdc/www/latex/subfigure.html
%%
\makeatletter
  \@ifpackageloaded{subfigure}{%

  %% Set up the subfigure package to italicize only the subfigure letters,
  %% not the parentheses, and to make \ref show only the bare letters,
  %% without any parentheses and without the main figure number.
  \let\p@subfigure\relax
  \renewcommand{\thesubfigure}{\emph{\alph{subfigure}}}
  \renewcommand*{\@thesubfigure}{(\thesubfigure)\hskip\subfiglabelskip}
  \renewcommand*{\@@thesubfigure}{(\thesubfigure)}


  %% Define a subfigure of any type with a caption underneath it
  %% Usage:   \sfobject[width]{item}{caption-text}{subfigure-label}
  %% Example: \sfobject{0.3\textwidth}{... a picture of some sort ...}{Above is a fancy picture}{fig}
  %% The subfigure-label must be unique for the whole document.
  %%
  \newcommand{\sfobject}[4]{%
    \subfigure[\label{sf:#4}#3]{\begin{minipage}[t]{#1}#2\end{minipage}}}
  \newcommand{\sfobjectuncaptioned}[2]{% Same spacing but no caption
    \subfigure{\begin{minipage}[t]{#1}#2\end{minipage}\addtocounter{subfigure}{-1}}}


  %% Define a subfigure consisting of a captioned \pdfpic
  %% Usage:   \sfpic{width}{filenamebase}{imagetype}{caption text}{subfigure-label}
  %% Example: \sfpic{0.3\textwidth}{picture}{jpg}{A picture}{renoir-pic}
  %%
  \newcommand{\sfpic}[5]{%
    \sfobject{#1}{\pdfpic{\textwidth}{!}{#2}{#3}}{#4}{#5}}%
  \newcommand{\sfpicuncaptioned}[3]{% Same but no caption
    \sfobjectuncaptioned{#1}{\pdfpic{\textwidth}{!}{#2}{#3}}}%

  %% Same as \sfpic but with a frame around the picture
  \newcommand{\sfpicframe}[5]{%
    \sfobject{#1}{\frame{\pdfpic{\textwidth}{!}{#2}{#3}}}{#4}{#5}}%
  \newcommand{\sfpicframeuncaptioned}[3]{%
    \sfobjectuncaptioned{#1}{\frame{\pdfpic{\textwidth}{!}{#2}{#3}}}}%


  %% Expand a reference to a subfigure, e.g. in caption or the main text
  %% \sfref{subfigure-label} yields (\emph{a}), etc.
  \newcommand{\sfref}[1]{(\ref{sf:#1})}


  %% Same as \sfref without the parentheses
  %% \sfrefnp{subfigure-label} yields \emph{a}, etc.
  \newcommand{\sfrefnp}[1]{\ref{sf:#1}}


  %% If producing overhead slides, call this command in the preamble
  %% so that (a), (b), etc. will be omitted from subfigure captions
  \newcommand{\sfstriplabels}{\renewcommand*{\@thesubfigure}{\mbox{}}}
}
\makeatother



%%%%%%%%%%%%%%%%%%%%%%%%%%%%%%%%%%%%%%%%%%%%%%%%%%%%%%%%%%%%%%%%%%%%%%%%%%%%%%%
%%  Misc

%% Puts a tiny note in the margin; useful for notes on drafts.
%%
%% To disable all of them uncomment the second def.
%% When you use it in the middle of a paragraph, you should follow 
%% the last brace with a comment character (%) to avoid getting
%% a double space in the output.
\newcommand{\draftnote}[1]{{\marginpar{\tiny\em\raggedright #1}}}
%%\renewcommand{\draftnote}[1]{}

%% The \draftnote command is based on \marginpar, which doesn't work 
%% in floating  bodies like figures and tables because the notes also
%% float.  So we provide a non-floating version of marginpar
%% (\rawmarginpar), and then use that to create a version of
%% \draftnote that works with figures (\rawdraftnote).  Note that
%% \rawdraftnotes might overlap, go off the page, etc.; they are not
%% juggled automatically like \draftnotes are.  \rawdraftnote
%% usually works best right before the figure caption.
%%
\newcommand{\rawmarginpar}[1]{%
\vadjust{\smash{\llap{\parbox[t]{0.7\marginparwidth}{#1}\kern\marginparsep}}}}
\newcommand{\rawdraftnote}[1]{{\rawmarginpar{\tiny\em\raggedright #1}}}

%% Pretends that its argument takes zero horizontal space, e.g. to 
%% prevent a caption from having a linebreak by letting it 
%% extend slightly into the space on either side.
%%
\providecommand{\zerospace}[1]{\mbox{\makebox[0in]{#1}}}

%% Same as \zerospace except that the result is left-
%% justified, as if it had zero length out to the right.
\providecommand{\zerolength}[1]{\mbox{\makebox[0in][l]{#1}}}


%% Centers the given object vertically and horizontally
%% within a square box of the specified size, ignoring 
%% the size of the object.  
%%
%% Usage: \squarebox[position]{width}{object}
%%
%% Examples:
%%   Center some text in the left 1/10 page and draw a border around it:
%%     \frame{\squarebox{0.1\textwidth}{Some text}}
%%   Center an image 1/8 the size of the page in a box twice its width:
%%     \frame{\squarebox{0.25\textwidth}{\pdfpic{0.5\textwidth}{!}{somefile}{pdf}}
\newcommand{\squarebox}[3][c]{%
    \begin{minipage}[#1]{#2}%
      \setlength{\unitlength}{\textwidth}%
      \begin{picture}(1.0,1.0)%
        \put(0.5,0.5){\makebox(0,0)[c]{#3}}%
      \end{picture}%
    \end{minipage}}

%% Command to add a degree symbol following a number
%%
%% When using it, you need to follow it with a "\ " if the next
%% character should be a space.  
%%
%% Examples: "23\degree\ line"    "the orientation was 23\degree."
%%
%% Used to be called \deg but there is now a LaTeX2e command by that
%% name. 
\newcommand{\degree}{\ensuremath{^\circ}}


%% Non-math-mode equivalents of math-mode characters, etc.
%% These are useful for preventing LaTeX2HTML from converting
%% tiny in-text expressions like these into GIFs, and can also 
%% make text a little more readable overall.
\newcommand{\superscript}[1]{\ensuremath{^#1}}
\newcommand{\subscript}[1]{\ensuremath{_#1}}
\newcommand{\mtimes}[2]{\ensuremath{#1 \times #2}}
\newcommand{\madd}[2]{\ensuremath{#1 + #2}}
\newcommand{\mplus}{\ensuremath{+}}
\newcommand{\mpm}{\ensuremath{\pm}}
\newcommand{\mlt}{\ensuremath{<}}
\newcommand{\mgt}{\ensuremath{>}}

%% Given a color type as argument, e.g. \psmono or \pscolor, returns the note 
%% that should be placed in the caption for that figure.
%% This will make figures that require color say so; I can't think of anything
%% useful for the rest to say.
%% Maybe \pscolororbw should say something like "(Figure looks better in color)"?
\newcommand{\figurenote}[1]{%
\ifx#1\pscolor{\ ({\em color figure\/})}\fi}


%% Dummy command marking a section of text as something for latex2html
%% to typeset as a single left-justified block  
\newcommand{\htmlblock}[1]{#1}


%% Constructs a titled figure caption.  Takes four arguments:
%%
%% \titledcaption[category]{label}{title}{captiontext}
%%
%% The "category" argument is optional; if present it should indicate the
%% type of image the file represents, i.e.  \psline for a line drawing,
%% \pscolor for a color photo, \pscolororbw for an image that is readable
%% either in color or monochrome, \psnone for a blank space, etc.  At
%% present, "category" is only used to see whether or not to
%% add a descriptive note for the title and list of figures entry.
%% See the \figurenote command for the cases where a note will be
%% added.  The default will never add any note.
%% 
%% The "label" is a name used to construct a label for the figure.
%%
%% The "title" should be a short heading, which will go in bold at the
%% top of the caption, and also goes into the list of figures.
%%
%% The "captiontext" will go into the caption only, not the list of
%% figures, after the title.  Just use an empty last argument {} if 
%% there is no caption text.
%% 
%% Complete figure example:
%% \begin{figure}
%%   \centering 
%%   \pspic[\psline]{\oneplotwidth}{!}{some-figure-filename-wo-ps-extension}
%%   \titledcaption[\psline]{fg:some-figure-label}
%%   {Title of the figure}{ 
%%     Some long text describing what this figure means and how it
%%     should be interpreted.}
%% \end{figure}
%%
%% jbednar040510: For some reason, I had replaced both "#1"s with 
%% "\protect{#1}", probably to fix some problem with the separate
%% versions for typesetters (below).  However, that broke the color
%% figure labels for the normal case, so I have removed it.  Next time
%% the typesetter versions are used, they will probably need to be
%% debugged.
%%
\newcommand{\titledcaption}[4][\psall]{%
    \caption[#3\figurenote{#1}]{%
      \label{#2}\mycapdefs%
      \textbf{\mathversion{bold}#3\figurenote{#1}.} %
      \htmlblock{#4}}}


%%%%%%%%%%%%%%%%%%%%%%%%%%%%%%%%%%%%%%%%%%%%%%%%%%%%%%%%%%%%%%%%%%%%%%%%%%%%%%%
%% Support for defining figure graphics separately from captions
%%
%% To use this, just pick a label you want to refer to the figure by,
%% and define the meat of the figure:
%%
%% \definefigbody{fg:some-fig}{%
%%  \sfpicframe{\twoplotwidth}{picture1}{jpg}{Case 1}{case-1-label}\hfill
%%  \sfpicframe{\twoplotwidth}{picture2}{jpg}{Case 2}{case-2-label}}
%%
%% Then wherever you want the figure to appear, insert it with a caption:
%%
%%  \insertfig[\pscolor]{fg:some-fig}{%
%%    Figure title}{%
%%    This figure conclusively demonstrates that...}%
%%
%% Leave off [\pscolor] if you don't want the "(color figure)" note
%% in the caption.
%%
%% Everything else will be handled automatically.  If you want fine
%% control over the figure (instead of automatically using
%% titledcaption, fullfigure*, etc.), just use 
%% \insertfigbody{fg:some-fig} to insert the meat of the figure into 
%% your own figure skeleton.
%%
%% If you also want to define the captions separately from where they
%% are first inserted into the text, then use \definefigcaption
%% instead of \insertfig.  That way the caption will be defined but
%% not yet inserted into the text.  Then just use \firstfigref
%% instead of \ref the first time you refer to the figure, and the
%% figure will automatically be placed into the text at that point.
%%
%% If you want the captioned figure to be placed somewhere other
%% than the first reference, just put an \insertfighere{fg-some-fig} 
%% command where you want the figure to appear.
%%
%%
%% Defines the body of the given figure
\newcommand{\definefigbody}[2]{%
  \ifundefined{figbody--#1}\expandafter\def\csname figbody--#1\endcsname{#2}%
  \else\errmessage{Figure #1 already defined}\fi}%
%%
%% Rarely used by end users:
\newcommand{\insertfigbody}[1]{\csname figbody--#1\endcsname}%
\newcommand{\insertfigtype}[1]{\csname figtype--#1\endcsname}%
\newcommand{\insertfigtitle}[1]{\csname figtitle--#1\endcsname}%
\newcommand{\insertfigcaption}[1]{\csname figcaption--#1\endcsname}%
%%
%% Defines the type, title, and caption of the given figure
\newcommand{\definefigcaption}[4][\psmono]{%
  \ifundefined{figtype--#2}\expandafter\def\csname figtype--#2\endcsname{#1}%
  \else\errmessage{Type for #2 already defined}\fi%
  \ifundefined{figtitle--#2}\expandafter\def\csname figtitle--#2\endcsname{#3}%
  \else\errmessage{Title for #2 already defined}\fi%
  \ifundefined{figcaption--#2}\expandafter\def\csname figcaption--#2\endcsname{#4}%
  \else\errmessage{Caption for #2 already defined}\fi}%
%%
%% Inserts the given figure given the type, title, and caption
\newcommand{\insertfig}[4][\psmono]{%
  \begin{fullfigure*}\insertfigbody{#2}\titledcaption[#1]{#2}{#3}{#4}\end{fullfigure*}}%
%%
%% Rarely used by end users:
\newcommand{\insertfighere}[1]{%
  \insertfig%
  [\insertfigtype{#1}]{#1}%
  {\insertfigtitle{#1}}%
  {\insertfigcaption{#1}}}%
%%
%% Refers to and inserts the given captioned figure
\newcommand{\firstfigref}[1]{\ref{#1}\insertfighere{#1}}%


%%%%%%%%%%%%%%%%%%%%%%%%%%%%%%%%%%%%%%%%%%%%%%%%%%%%%%%%%%%%%%%%%%%%%%%%%%%%%%%
%% Support for figures at end of document for journal paper
%% submissions
%% 
%% None of these commands are used in a normal camera-ready paper
%%
%% For these commands to work, you should use \firstfigref
%% consistently (which allows the actual caption to be inserted 
%% wherever).  
%%
%% If you use subfigures, you should also add a line:
%%   \PassOptionsToPackage{normalsize}{subfigure}
%% before the \usebpackage{subfigure}, to make the fonts the right 
%% size.


%% Declare that figures should be put at end of paper for typesetting
\newcommand{\figuresatend}[1]{%
  \renewcommand{\firstfigref}[1]{\begin{center}--- Insert
      figure~\ref{##1} about here ---\end{center}\ref{##1}}%
  %% Hack to make subfigure package work outside of a float environment
  \makeatletter\newcommand{\@captype}{figure}\makeatother%
}

%% Override this if you need to add your name or manuscript number
%% Accepts one argument, the figure key (suitable for \ref)
\newcommand{\submissionfigdescription}[1]{Figure~\ref{#1}}

% Format a figure for typesetting
% Usage: \insertsubmissionfig[separation]{reference}
\newcommand{\insertsubmissionfig}[2][1in]{%
\newpage\setcounter{subfigure}{0}
\centering
\makebox[0in]{%
\resizebox{7.5in}{!}
{\begin{minipage}{\textwidth}
    \centering%
    \insertfigbody{#2}\\[#1]
    \submissionfigdescription{#2}
  \end{minipage}}}%
}


% Format a caption for typesetting
% Usage: \insertsubmissioncaption{reference}
\newcommand{\insertsubmissioncaption}[1]{%
  \newpage%
  \begin{figure}[h]%
    \renewcommand{\baselinestretch}{1.5}\small\normalsize
    \caption[\insertfigtitle{#1}\figurenote{\protect{\insertfigtype{#1}}}]{%
      \label{#1}%
      \textbf{\insertfigtitle{#1}\figurenote{\protect{\insertfigtype{#1}}}}.
      \raggedright%
      \htmlblock{\insertfigcaption{#1}}}
  \end{figure}%
}



%%%%%%%%%%%%%%%%%%%%%%%%%%%%%%%%%%%%%%%%%%%%%%%%%%%%%%%%%%%%%%%%%%%%%%%%%%%%%%%
%% Tech report title pages
%%
%% Default values; you can (and should) override these using
%% \renewcommand within your own paper.  Often you will need to add
%% different linebreaks to the \techreporttitle and
%% \techreportauthor since they need to fit in such a narrow window.
%%
\newcommand{\techreporttitle}{\title}
\newcommand{\techreportauthor}{\author}
\newcommand{\techreportnumber}{}
\newcommand{\techreportemailaddress}{}
\newcommand{\techreporthttpaddress}{http://www.cs.utexas.edu/users/nn}
\newcommand{\techreportdepartment}{Artificial Intelligence Laboratory}
\newcommand{\universityname}{The University of Texas at Austin}
\newcommand{\universityaddress}{Austin, TX 78712}

%% Call just after your \begin{document} to actually generate the tech
%% report title page.
%%
%% Takes two arguments -- the offset in x and y to correct for the
%% left and top margins, respectively, in the current style.  The
%% values should both be zero for the default 1-inch margins in
%% nnonecolumn.sty; adjust other margins as needed.
%%
\newcommand{\techreporttitlepage}[2]{%
  \newpage
  \thispagestyle{empty}
  \setlength{\unitlength}{1in}%
  \begin{picture}(0,0)(#1,#2)
    \put(1,-2.3){\parbox{4in}{%
      \begin{center}%
        {\textbf{\Large \techreporttitle}}\\[8mm]
        {\textbf{\large \techreportauthor}}\\[8mm]
        \textbf{\large \techreportnumber}
      \end{center}}}
    \put(1,-7.8){\parbox{4in}{%
      \begin{center}%
        {\tt \techreportemailaddress}\\
        {\tt \techreporthttpaddress}\\[8mm]
        \techreportdepartment\\
        \universityname\\
        \universityaddress
      \end{center}}}
  \end{picture}
  \newpage
}


%%%%%%%%%%%%%%%%%%%%%%%%%%%%%%%%%%%%%%%%%%%%%%%%%%%%%%%%%%%%%%%%%%%%%%%%%%%%%%%
% For emacs:
%Local Variables:
%TeX-command-default:"Check"
%End:


%% Determine which pictures to include, e.g. for faster viewing (optional)
%\psdrafttype{\psmono} %Uncomment to turn off color images
%\psdrafttype{\psline} %Uncomment to turn off bitmap images
%\psdrafttype{\psnone} %Uncomment to turn off all images for %super-fast viewing
%\psdrafttype{\psall}  %Default: show all images

%% Override the standard picture width to be smaller for two-column mode
\setlength{\oneplotwidth}{0.46\textwidth} 

%% Uncomment this to get a one-column draft with lots of space for
%% comments (great for giving to a coauthor for edits)
%\scribbledraft

\title{Evolving Modular Controls for Game Agents
%% Here is one way of putting a reference on the first page, but it 
%% messes up the page layout so it should not be used in cameraready
%% papers (see the latex-to-appear.html page in the NN User's Guide):
%%\thanks{To appear in the {\em Proceedings of the 19th Annual
%%Conference of the Cognitive Science Society (CogSci-97, Stanford CA)}, 1997.}
} 
 
\author{
  {\large\bf \htmladdnormallink{James~A.~Bednar}{http://www.cs.utexas.edu/users/jbednar/}
    ({\tt\htmladdnormallink{jbednar@cs.utexas.edu}{mailto:jbednar@cs.utexas.edu}})}\\
  {\large\bf \htmladdnormallink{Risto~Miikkulainen}{http://www.cs.utexas.edu/users/risto/}
    ({\tt\htmladdnormallink{risto@cs.utexas.edu}{mailto:risto@cs.utexas.edu}})} \\
  \htmladdnormallink{Department of Computer Sciences}{http://www.cs.utexas.edu/}\\
  \htmladdnormallink{University of Texas at Austin}{http://www.utexas.edu/}\\
  Austin, TX 78712 USA
  %%\\
  %%Technical Report AI97-555\\
  %%August 1997
  %%
  } 

\date{}

\begin{document} 

\setlength{\baselineskip}{11pt}

\maketitle 
%%
%% Use this to make the document double-spaced, like in journal submissions
%%\renewcommand{\baselinestretch}{1.5}\small\normalsize\setlength{\footnotesep}{1.2em}

%% Comment these lines out for class reports, journal submissions, etc.
%% which don't need the to-appear information and separating line.
\thispagestyle{fancy}
\chead{\footnotesize Appears in the {\em Proceedings of the 19th Annual
    Conference of the Cognitive Science Society (CogSci-97, Stanford CA)}, 1997.}
\cfoot{}
%%\renewcommand{\headrulewidth}{0pt}   % Enables or disables line after header
%%\renewcommand{\footrulewidth}{0.5pt} % Enables or disables line before footer


\begin{abstract} 
  RF-LISSOM, a self-organizing model of laterally connected
  orientation maps in the primary visual cortex, was used to study the
  psychological phenomenon known as the tilt aftereffect.  The same
  self-organizing processes that are responsible for the long-term development
  of the map and its lateral connections are shown to result in tilt
  aftereffects over short time scales in the adult.  The model allows
  observing large numbers of neurons and connections simultaneously,
  making it possible to relate higher-level phenomena to low-level
  events, which is difficult to do experimentally.  The results give
  computational support for the idea that direct tilt 
  aftereffects arise from adaptive lateral interactions between
  feature detectors, as has long been surmised.  They also suggest
  that indirect effects could result from the conservation of synaptic
  resources during this process.  The model thus provides a unified
  computational explanation of self-organization and both direct and
  indirect tilt aftereffects in the primary visual cortex.
\end{abstract} 


 
%%%%%%%%%%%%%%%%%%%%%%%%%%%%%%%%%%%%%%%%%%%%%%%%%%%%%%%%%%%%%%%%%%%%%%%%%%%%%%%
\section{Introduction}

The tilt aftereffect (TAE,~\npcite{gibson:adaptation}) is a simple but
intriguing visual phenomenon.  After staring at a pattern of tilted
lines or gratings, subsequent lines appear to have a slight tilt in
the opposite direction (Figure~\ref{fg:aftereffect-patterns}).  The
effect resembles an afterimage from staring at a bright light, but it
reflects changes in orientation perception rather than in color or
brightness.

%% [Old location of~\ref{fg:aftereffect-patterns}.]

Most modern explanations of the TAE are based on the {\em
  feature-detector\/} model of the visual cortex \cite{hubel:monkey}.
Individual orientation detectors become more difficult to excite
during repeated presentation of oriented stimuli, and the
desensitization persists for some time afterwards.  This observation
forms the basis of the {\em fatigue\/} theory of the TAE: if active
neurons become fatigued over time, the set of neurons activated for a
test figure will shift away from the adaptation orientation.  Assuming
the perceived orientation is some sort of average over the orientation
preferences of the activated neurons, the perceived orientation would
thus show the direct TAE \cite{coltheart:psyrev71}.

The fatigue theory has been discredited because it has become apparent
that the adaptation is mediated by the lateral connections between
neurons, rather than changes occurring within the neurons themselves
\cite{bednar:aitr97,vidyasagar:neurosci90}.  The now-popular {\em
  inhibition\/} theory postulates that tilt aftereffects result from
changing inhibition between neurons~\cite{tolhurst:vres75}, perhaps by
increases in the strength of lateral connections between them.

%% Figure was placed here so that it would not float to the top of the
%% first column, since it would then be above the abstract.
\begin{figure}
  \centering 
  \pdfpic{\oneplotwidth}{!}{ps/aftereffects_demonstration_with_indirect}{pdf}
  \titledcaption[\psline]{fg:aftereffect-patterns}
  {Tilt aftereffect patterns}
  {Fixate your gaze upon the circle inside the square at the center
    for at least thirty seconds, moving your eye slightly inside the
    circle to avoid developing strong afterimages.  Now fixate upon
    the figure at the left.  The vertical lines should appear slightly
    tilted to the right; this phenomenon is called the direct tilt
    aftereffect.  If you fixate upon the horizontal lines at the
    right, they should appear barely tilted counterclockwise, 
    demonstrating the indirect tilt aftereffect.
    (Adapted from~\npcite{campbell:vres71}.)  
    %\vspace{-\baselineskip}% Make figure a little tighter
    }
\end{figure}

Although the inhibition theory was first proposed in the 1970s, only recently
has it become computationally feasible to test in a detailed
model of cortical function.  A Hebbian self-organizing process (the
Receptive-Field Laterally Interconnected Synergetically
Self-Organizing Map, or RF-LISSOM; \npcite*{miikkulainen:psylm97,sirosh:phd,sirosh:bc94,sirosh:npl96,sirosh:neuralcomp,sirosh:htmlbook96-article})
has been shown to develop feature detectors and specific lateral
connections that could produce such aftereffects.  The
RF-LISSOM model gives rise to anatomical and functional
characteristics of the cortex such as topographic maps, ocular
dominance, orientation, and size preference columns, and the patterned
lateral connections between them.  Although other models exist that
explain how the feature-detectors and afferent connections could
develop by input-driven self-organization, RF-LISSOM is the only model
that also shows how the lateral connections can self-organize as an
integral part of the process.  The laterally connected model has also
been shown to account for many of the dynamic aspects of the visual
cortex, such as reorganization following retinal and cortical lesions
(\npcite{miikkulainen:psylm97,sirosh:phd,sirosh:cns94};
\npcite*{sirosh:htmlbook96-article}). 

The current work is a first study of the {\em functional\/} behavior
of the model, specifically the response to stimuli similar to those
known to cause the TAE in humans.  The RF-LISSOM model allows
observing activation and connection patterns between large
numbers of neurons simultaneously, making it possible to relate
higher-level phenomena to low-level events, which is difficult to do
experimentally.  The results suggest that tilt aftereffects are not
flaws in an otherwise well-designed system, but an unavoidable result
of a self-organizing process that aims at producing an efficient,
sparse encoding of the input through decorrelation (as proposed by
\npcite{barlow:aftereffects}; see also
\npcite{dong:decorrelation,field:goal,foldiak:bc90,miikkulainen:psylm97}; 
\npcite*{sirosh:htmlbook96-article}).



%%%%%%%%%%%%%%%%%%%%%%%%%%%%%%%%%%%%%%%%%%%%%%%%%%%%%%%%%%%%%%%%%%%%%%%%%%%%%%%
\section{Architecture}

\begin{figure}
  \centering
  \begin{minipage}{\oneplotwidth}
    \vspace{0.1in} %% Correct for bogus bounding box in eps file
    \centering
    \pdfpic{\textwidth}{!}{ps/rf-lissom-architecture-bw}{pdf}
  \end{minipage}
  \titledcaption[\pscolororbw]{fg:rf-lissom-architecture} 
  {Architecture of the RF-LISSOM network}
  {A tiny RF-LISSOM network and retina are shown, along with
    connections to a single neuron (shown as a large circle).  The
    input is an oriented Gaussian activity pattern on the retinal
    ganglion cells.  The afferent connections form a local anatomical
    receptive field on the simulated retina.  Neighboring neurons have
    different but highly overlapping RFs.  Each neuron computes an
    initial response as a dot product of its receptive field and its
    afferent weight vector. The responses then repeatedly propagate
    within the cortex through the lateral connections and evolve into
    an activity ``bubble''. After the activity stabilizes, weights of
    the active neurons are adapted. 
    }
  \vspace{-0.1in} %% Correct for bogus bounding box in eps file
\end{figure}

The cortical architecture for the model has been simplified and
reduced to the minimum necessary configuration to account for the
observed phenomena.  Because the focus is on the two-dimensional
organization of the cortex, each ``neuron'' in the model cortex
corresponds to a vertical column of cells through the six layers of
the human cortex.  
%%
The cortical network is modeled with a sheet of interconnected neurons
and the retina with a sheet of retinal ganglion cells
(figure~\ref{fg:rf-lissom-architecture}).  Neurons receive afferent
connections from broad overlapping patches on the retina. The 
\mtimes{N}{N} network is projected on to the retina of \mtimes{R}{R}
ganglion cells, and each neuron is connected to ganglion cells in a
circular area of radius $r$ around the projections. Thus, neurons at a
particular cortical location receive afferents from the corresponding
location on the retina.  Since the LGN accurately reproduces the
receptive fields of the retina, it has been bypassed for simplicity.

Each neuron also has reciprocal excitatory and inhibitory lateral
connections with itself and other neurons.  Lateral excitatory
connections are short-range, connecting each neuron with itself and
its close neighbors.  Lateral inhibitory connections run for
comparatively long distances, but also include connections to the
neuron itself and to its neighbors.

The input to the model consists of 2-D ellipsoidal Gaussian patterns
representing retinal ganglion cell activations.  For training, the
orientations of the Gaussians are chosen randomly from the uniform
distribution in the range $[0,\pi)$.  The elongated spots approximate
natural visual stimuli after the edge detection and enhancement
mechanisms in the retina.  They can also be seen as a model of the
intrinsic retinal activity waves that occur in late pre-natal
development in mammals \cite*{meister:synchronous}.  The RF-LISSOM
network models the self-organization of the visual cortex based on
these natural sources of elongated features.

The afferent weights are initially set to random values, and the
lateral weights are preset to a smooth Gaussian profile.  The
connections are organized through 
an unsupervised learning process. At each training step, neurons start
out with zero activity.  The initial response $\eta_{ij}$ of neuron
$(i,j)$ is calculated as a weighted sum of the retinal activations:

\begin{equation}
  \label{eq:initial}
  \eta_{ij} = \sigma \left( \sum_{a,b} \xi_{ab} \mu_{ij,ab} \right),
\end{equation} 

where $\xi_{ab}$ is the activation of retinal ganglion $(a,b)$ within
the anatomical RF of the neuron, $\mu_{ij,ab}$ is the corresponding
afferent weight, and $\sigma$ is a piecewise linear approximation of
the sigmoid activation function.  
%%
The response evolves over a very short time scale through lateral
interaction.  At each time step, the neuron combines the above
afferent activation $\sum \xi \mu$ with lateral excitation and
inhibition:

%% This is quite ugly, both in the source code and typeset versions,
%% but it was necessary to make it fit into the narrow two-column
%% format.  The htmlonly version (from the master's thesis) is 
%% better if it fits.
\begin{latexonly}
  \newcommand{\parenheight}{\rule[-3.5ex]{0in}{7ex}}
  \newcommand{\columnadjust}{\hspace{-2em}}
  \begin{eqnarray}
    \label{eq:biollat}
    \eta_{ij}(t) = \sigma \left( \parenheight \right.
    & & \columnadjust \sum \xi \mu  +  \gamma_e \sum_{k,l} E_{ij,kl} \eta_{kl}(t-1) - \nonumber \\
    & & \columnadjust \gamma_i  \sum_{k,l} I_{ij,kl} \eta_{kl}(t-1) \left. \parenheight \right) ,
  \end{eqnarray}
\end{latexonly}

\begin{htmlonly}
  \begin{equation}
    \label{eq:biollat}
    \eta_{ij}(t) = \sigma \left(   \sum \xi \mu +
      \gamma_e \sum_{k,l} E_{ij,kl} \eta_{kl}(t-1) -
      \gamma_i \sum_{k,l} I_{ij,kl} \eta_{kl}(t-1) \right) ,
  \end{equation}
\end{htmlonly}

where $E_{ij,kl}$ is the excitatory lateral connection weight on the
connection from neuron $(k,l)$ to neuron $(i,j)$, $I_{ij,kl}$ is the
inhibitory connection weight, and $\eta_{kl}(t-1)$ is the activity of
neuron $(k,l)$ during the previous time step.  
%%
The scaling factors $\gamma_e$ and $\gamma_i$ determine 
the relative strengths of excitatory and inhibitory lateral
interactions.  

While the cortical response is settling, the retinal
activity remains constant.
%%
The activity pattern starts out diffuse and spread over a substantial
part of the map, but within a few iterations of
equation~\ref{eq:biollat}, converges into a small number of stable
focused patches of activity, or activity bubbles.  
%%
After the activity has settled, the connection weights of each neuron
are modified. Both afferent and lateral weights adapt according to the
same mechanism: the Hebb rule, normalized so that the sum of the
weights is constant:

\begin{equation}
  \label{eq:firstdw}
  w_{ij,mn}(t+\delta t)=\frac{ w_{ij,mn}(t) + \alpha \eta_{ij} X_{mn}}
  {\sum_{mn} \left[ w_{ij,mn}(t) + \alpha \eta_{ij} X_{mn} \right]},
\end{equation}

where $\eta_{ij}$ stands for the activity of neuron $(i,j)$ in the
final activity bubble, $w_{ij,mn}$ is the afferent or lateral
connection weight ($\mu$, $E$ or $I$), $\alpha$ is the learning rate
for each type of connection ($\alpha_A$ for afferent weights,
$\alpha_E$ for excitatory, and $\alpha_I$ for inhibitory) and $X_{mn}$
is the presynaptic activity ($\xi$ for afferent, $\eta$ for lateral).
%%
The larger the product of the pre- 
and post-synaptic activity $\eta_{ij} X_{mn}$, the larger the weight
change. Therefore, when the pre- and post-synaptic neurons fire
together frequently, the connection becomes stronger.  Both excitatory
and inhibitory connections strengthen by correlated activity;
normalization then redistributes the changes so that the sum of each
weight type for each neuron remains constant.  

At long distances, very few neurons have correlated activity and
therefore most long-range connections eventually become weak.  The
weak connections can be eliminated periodically, resulting in patchy
lateral connectivity similar to that observed in the visual cortex.
%%
The radius of the lateral excitatory interactions starts out large,
but as self-organization progresses, it is decreased until it covers
only the nearest neighbors. Such a 
decrease is necessary for global topographic order to develop and for
the receptive fields to become well-tuned at the same time.



%%%%%%%%%%%%%%%%%%%%%%%%%%%%%%%%%%%%%%%%%%%%%%%%%%%%%%%%%%%%%%%%%%%%%%%%%%%%%%%
\section{Experiments}

The model consisted of an array of \mtimes{192}{192} neurons, and a
retina of \mtimes{24}{24} ganglion cells. The circular anatomical
receptive field of each neuron was centered in the portion of the
retina corresponding to the location of the neuron in the cortex.  The
RF consisted of random-strength connections to all ganglion cells less
than 6 units away from the RF center.  
%%
The cortex was self-organized for $30,000$ iterations on oriented
Gaussian inputs with major and minor axes of half-width $\sigma=7.5$
and $1.5$, respectively.\footnote{ The initial lateral excitation radius
  was $19$ and was gradually decreased to $1$.  The lateral inhibitory
  radius of each neuron was $47$, and inhibitory connections whose
  strength was below $0.00025$ were pruned away at $30,000$
  iterations.  The lateral inhibitory connections were preset to a
  Gaussian profile with $\sigma=100$, and the lateral excitatory
  connections to a Gaussian with $\sigma=15$.  The lateral excitation
  $\gamma_e$ and inhibition strength $\gamma_i$ were both $0.9$. The
  learning rate $\alpha_{\rm A}$ was gradually decreased from $0.007$ to
  $0.0015$, $\alpha_{\rm E}$ from $0.002$ to $0.001$ and $\alpha_{\rm
    I}$ was a constant $0.00025$.  The lower and upper thresholds of
  the sigmoid were increased from $0.1$ to $0.24$ and from $0.65$ to
  $0.88$, respectively.  The number of iterations for which the
  lateral connections were allowed to settle at each training
  iteration was initially $9$, and was increased to $13$
  over the course of training.  The parameter settings were identical
  to those of \emcite{sirosh:phd}, and were not tuned or tweaked for
  the tilt aftereffect simulations.  Small variations produce roughly
  equivalent results \cite{sirosh:phd}.   
} 
The training took 8 hours on 64 processors of a Cray T3D at the
Pittsburgh Supercomputing Center.  The model requires more than
three gigabytes of physical memory to represent the more than 400
million connections in this small section of the cortex.

\subsection{Orientation map organization}

In the self-organization process, the neurons developed oriented
receptive fields organized into orientation columns very similar to
those observed in the primary visual cortex.
%%
The strongest lateral connections of highly-tuned cells 
link areas of similar orientation preference, and avoid neurons with
the orthogonal orientation preference.
%%
Furthermore, the connection patterns of highly oriented neurons are
typically elongated along the direction in the map that corresponds to
the neuron's preferred stimulus orientation. This organization
reflects the activity correlations caused by the elongated Gaussian
input pattern: such a stimulus activates primarily those neurons that
are tuned to the same orientation as the stimulus, and located along
its length \cite*{sirosh:htmlbook96-article}.
%%
Since the long-range lateral connections are inhibitory, the net
result is {\em decorrelation\/}: redundant activation is removed,
resulting in a sparse representation of the novel features of each
input (\npcite{barlow:aftereffects,field:goal};
\npcite*{sirosh:htmlbook96-article}). 
As a side effect, illusions and aftereffects may sometimes occur, as
will be shown below.


\subsection{Aftereffect simulations}

In psychophysical measurements of the TAE, a fixed stimulus is
presented at a particular location on the retina.  To simulate these
conditions in the model, the position and angle of the inputs were
fixed to a single value for a number of iterations, rather than having
a uniform random distribution as in self-organization.  To permit more
detailed analysis of behavior at short time scales, the learning rates
were reduced from those used during self-organization, to
$\alpha_A=\alpha_E=\alpha_I=0.00005$.  All other parameters remained
as in self-organization.

\begin{figure}
  %% Taken from the master's thesis.  See that file for more info.
  \centering
  \pdfpic{\oneplotwidth}{!}{ps/970101_ae_090d_avg.setl.compared_to_MM76}{pdf}
  \titledcaption[\psline]{fg:tae-vs-angle}
  {Tilt aftereffect versus retinal angle}
  {The open circles represent the average tilt aftereffect for a
    single human subject (DEM) from \emcite{mitchell:vres76} over ten
    trials.  For each angle in each trial, the subject adapted for
    three minutes on a sinusoidal grating of a given angle, then was
    tested for the effect on a horizontal grating. Error bars indicate
    \mpm 1 standard error of measurement.  The subject shown had the
    most complete data of the four in the study.  All four showed very
    similar effects in the x-axis range \mpm 40\degree; the
    indirect TAE for the larger angles varied widely between
    \mpm 2.5\degree.  The graph is roughly anti-symmetric around
    0\degree, so the TAE is essentially the same in both directions
    relative to the adaptation line. 
    %%
    The heavy line shows the average magnitude of the tilt aftereffect
    in the RF-LISSOM model over nine trials at different locations on
    the retina.  Error bars indicate \mpm 1 standard error of
    measurement.  The network adapted to a vertical adaptation line at
    a particular position for 90 iterations, then the TAE was measured
    for test lines oriented at each angle.  The duration of adaptation
    was chosen so that the magnitude of the human data and the model
    match; this was the only parameter fit to the data.  
    %% The shape is constant, only the mag changes...
    The result from the model
    closely resembles the curve for humans at all angles, showing both
    direct and indirect tilt aftereffects.
    %\vspace{-\baselineskip}% Make figure a little tighter
    }
\end{figure}

To compare with the psychophysical experiments, perceived
orientations were compared before and after tilt adaptation.
Perceived orientation was measured as a vector sum over all active
neurons, with the magnitude of each vector representing the activation
level, and the vector direction representing the orientation
preference of the neuron before adaptation.  Perceived orientation was
computed separately for each possible orientation of the test
Gaussian, both before and after adaptation.  For a given angular
separation of the adaptation stimulus and the test stimulus, the
computed magnitude of the tilt aftereffect is the difference between
the initial perceived angle and the one perceived after adaptation.
Figure~\ref{fg:tae-vs-angle} plots these differences after adaptation
for 90 iterations of the RF-LISSOM algorithm.  For comparison,
figure~\ref{fg:tae-vs-angle} also shows the most detailed data
available for the TAE in human foveal vision \cite{mitchell:vres76}.

The results from the RF-LISSOM simulation are strikingly similar to 
the psychophysical results.  For the range 5\degree\ to 40\degree, all
subjects in the human study (including the one shown) exhibited angle
repulsion effects nearly identical to those found in the RF-LISSOM
model.  The magnitude of this {\em direct\/} TAE increases very rapidly
to a maximum angle repulsion at approximately 10\degree, falling off
somewhat more gradually to zero as the angular separation increases.

The results for larger angular separations (from 45\degree\ to 85\degree)
show a greater inter-subject variability in the psychophysical
literature, but those found for the RF-LISSOM model are well within
the range seen for human subjects.  The {\em indirect\/} effects for
the subject shown were typical for that study, although some subjects
showed effects up to 2.5\degree.  


In addition to the angular changes in the TAE, its magnitude in humans
increases regularly with adaptation time \cite{gibson:adaptation}.
The equivalent of ``time'' in the RF-LISSOM model is an iteration,
i.e.\ a single cycle of input presentation, activity propagation,
settling, and weight modification.  As the number of adaptation
iterations is increased, the magnitude of the TAE in the model
increases monotonically, while retaining the same basic shape of
figure~\ref{fg:tae-vs-angle} \cite{bednar:aitr97}.  The curve
that best matches the human data was shown in
figure~\ref{fg:tae-vs-angle}. 

Due to the time required to obtain even a single point on the angular
curve of the TAE for human subjects, complete experimental
measurements of the angular function at different adaptation times are
not available.  However, when the time course of the direct TAE is
measured at a single orientation, the increase is approximately
logarithmic with time \cite{gibson:adaptation}, eventually saturating
at a level that depends upon the experimental protocol used
\cite{greenlee:vres87sat,magnussen:vres86}.
Figure~\ref{fg:tae-vs-time-for-humans} compares the shape of this TAE
versus time curve for human subjects and for the RF-LISSOM model.
%%
%% This is some gratuitously wrapped text around a figure that has
%% been squished a bit (to be of a shape worth wrapping text around).   
%% Such squashing is not recommended, nor is such wrapping likely to
%% be useful in a two-column paper like this one, but this shows what
%% you can do with parpic/picins.
\begin{figure*}
  \centering
  %% The first two arguments set the size of the box to leave for the
  %% graphic; often they will need adjustment to get an appropriate
  %% amount of whitespace around it.
  \parpic(0.6\textwidth,0.37\textwidth)[r][rt]{%
    \pdfpic{0.6\textwidth}{0.3\textwidth}{ps/970101_ae_090d_avg.setl_wrt_time.compared_to_GM87}{pdf}}
  %% I'm not sure why, but this adjustment is needed for the top
  %% margin to line up with other pages:
  \vspace*{-2.6ex}
  \titledcaption[\psline]{fg:tae-vs-time-for-humans}
  {Direct tilt aftereffect versus time}
  {The circles show the magnitude of the TAE as a function of
    adaptation time for human subjects MWG (unfilled circles) and SM
    (filled circles) from \emcite{greenlee:vres87sat}; they were the
    only subjects tested in the study.  Each subject adapted to a
    single \mplus 12\degree line for the time period indicated on the
    horizontal axis (bottom).  To estimate the magnitude of the
    aftereffect at each point, a vertical test line was presented at
    the same location and the subject was requested to set a
    comparison line at another location to match it.  The plots
    represent averages of five runs; the data for 0 -- 10 minutes were
    collected separately from the rest.
    %%
    For comparison, the heavy line shows average TAE in the LISSOM 
    model for a \mplus 12\degree test line over 9 trials (with parameters as in
    figure~\ref{fg:tae-vs-angle}).  The horizontal axis (top)
    represents the number of iterations  of adaptation, and the
    vertical axis represents the magnitude of the TAE at this time
    step.
    %%
    The RF-LISSOM results show a similar logarithmic increase in TAE
    magnitude with time, but do not show the saturation that is seen
    for the human subjects.}
\end{figure*}%%
%%%%
%%%%
%%%%
%%%%    %% Normal version of this figure (without picins)
%%%%    \begin{figure}
%%%%      \centering
%%%%      \pdfpic{\oneplotwidth}{!}{ps/970101_ae_090d_avg.setl_wrt_time.compared_to_GM87}{pdf}
%%%%      \titledcaption[\psline]{fg:tae-vs-time-for-humans}
%%%%      {Direct tilt aftereffect versus time}
%%%%      {The circles show the magnitude of the TAE as a function of
%%%%        adaptation time for human subjects MWG (unfilled circles) and SM
%%%%        (filled circles) from \emcite{greenlee:vres87sat}; they were the
%%%%        only subjects tested in the study.  Each subject adapted to a
%%%%        single \mplus 12\degree line for the time period indicated on the
%%%%        horizontal axis (bottom).  To estimate the magnitude of the
%%%%        aftereffect at each point, a vertical test line was presented at
%%%%        the same location and the subject was requested to set a
%%%%        comparison line at another location to match it.  The plots
%%%%        represent averages of five runs; the data for 0 -- 10 minutes were
%%%%        collected separately from the rest.
%%%%        %%
%%%%        For comparison, the heavy line shows average TAE in the LISSOM 
%%%%        model for a \mplus 12\degree test line over 9 trials (with parameters as in
%%%%        figure~\ref{fg:tae-vs-angle}).  The horizontal axis (top)
%%%%        represents the number of iterations  of adaptation, and the
%%%%        vertical axis represents the magnitude of the TAE at this time
%%%%        step.
%%%%        %%
%%%%        The RF-LISSOM results show a similar logarithmic increase in TAE
%%%%        magnitude with time, but do not show the saturation that is seen
%%%%        for the human subjects.
%%%%    }
%%%%    \end{figure}%%
%%%%
The $x$ axis for the RF-LISSOM and human data has different units, but
the correspondence between the two curves might provide a rough way of
quantifying the equivalent real time for an ``iteration'' of the
model.  The time course of the TAE in the RF-LISSOM model is similar
to the human data.  The TAE increases approximately logarithmically,
but it does not completely saturate over the adaptation amounts tested
so far.  This difference suggests that the biological implementation
has additional constraints on the amount of learning that can be
achieved over the time scale over which the tilt aftereffect is seen.


\subsection{How does the TAE arise in the model?}

The TAE seen in figures~\ref{fg:tae-vs-angle}
and~\ref{fg:tae-vs-time-for-humans} must result from changes 
in the connection strengths between neurons, since no other
component of the model changes as adaptation progresses.
Simulations performed with only one type of weight (either afferent,
lateral excitatory, or lateral inhibitory) adapting at a given time
show that the inhibitory weights determine the shape of the curve for
all angles \cite{bednar:aitr97}.  The small component of the TAE
resulting from adaptation of either type of excitatory weights is
almost precisely opposite the total effect.
%%
Although each inhibitory connection adapts with the same learning rate
as the excitatory connections ($\alpha_I=\alpha_A=\alpha_E=0.00005$),
there are many more inhibitory connections than excitatory
connections.  
%%
The combined strength of all the small inhibitory changes outweighs
the excitatory changes, and results in a curve with a sign opposite
that of the components from the excitatory weights. 

In what way do the changing inhibitory connections cause these
effects?  During adaptation, we see that the response to the 0\degree\ 
adaptation line becomes gradually more concentrated towards the
central area of the Gaussian pattern presented.  This is because the
inhibition between active neurons increases, allowing only the most
strongly activated neurons to remain active after settling
(equation~\ref{eq:biollat}).  However, the distribution of active
orientation detectors is centered around the same angle, so 
the same angle is perceived.
  
The response to a test line with a slightly different orientation
(e.g.\ 10\degree) is also more focused after adaptation, but the overall
distribution of activated neurons has shifted.  Fewer neurons that
prefer orientations close to the adaptation line now respond, but an
increased number of those that prefer distant angles do.  This is
because inhibition was strengthened primarily between neurons close to
the adaptation angle, and not between those which prefer larger
orientations, greater than the 10\degree\ test line.  The net effect is a
shift of the perceived orientation {\em away\/} from the adaptation
angle, resulting in the direct TAE\@.

In contrast, the response to a very different test line (e.g.\ 60\degree)
is broader and stronger after adaptation.  Adaptation occurred only in
activated neurons, so neurons with orientation preferences greater
than 60\degree are unchanged.  However, those with preferences somewhat
less than 60\degree\ actually now respond more strongly.  During
adaptation, their inhibitory connections with other active neurons,
i.e.\ those that represent orientations close to the 0\degree\ adaptation line, became
stronger.  Since the sum of inhibition is constant for each neuron
(equation~\ref{eq:firstdw}), the connections to neurons representing
distant angles (e.g.\ 60\degree) became weaker.  As a result, the 60\degree\ 
line now inhibits them less than before adaptation.  Thus they are
more active, and the perceived orientation has shifted towards 0\degree.
This indirect effect is therefore true to its name, caused indirectly
by the strengthening of inhibitory connections.  The RF-LISSOM model
thus shows computationally that both the direct and indirect effects
could be caused by activity-dependent adaptation of inhibitory lateral
interactions. 


%%%%%%%%%%%%%%%%%%%%%%%%%%%%%%%%%%%%%%%%%%%%%%%%%%%%%%%%%%%%%%%%%%%%%%%%%%%%%%%
\section{Discussion and Future Work}


The results presented above suggest that the same self-organizing
principles that result in sparse coding and reduce redundant
activation may also be operating over short time intervals in the
adult, with quantifiable psychological consequences such as the TAE\@.
This finding demonstrates a potentially important computational link
between development, structure, and function.

Even though the RF-LISSOM model was not originally developed as an
explanation for the tilt aftereffect, it exhibits tilt aftereffects
that have nearly all of the features of those measured in humans.  The
effect of varying angular separation between the test and adaptation
lines is similar to human data at all orientations, the time course is
approximately logarithmic in each, and the TAE is localized to the
retinal location which experienced the stimulus.  With minor
extensions, the model should account for other features of the TAE,
such as higher variance at oblique orientations, frequency
localization, movement direction specificity, and ocular transfer.
For a discussion of the match between the model and data for humans
from a variety of experiments, see \emcite{bednar:aitr97}.

The only prominent features of the TAE that do not directly follow
from the model are saturation of the effect for long adaptations, and
recovery of accurate perception even in complete darkness
\cite{greenlee:vres87sat,magnussen:vres86}.  These two features
suggest that the inhibitory weights modified during tilt adaptation
could actually be a set of small, temporary weights adding to or
multiplying more permanent connections.  Such a mechanism was proposed
by \emcite{vondermalsburg:synaptic} as an explanation of visual object
segmentation; this idea was implemented for the RF-LISSOM model by
\emcite{choe:utcstr96} and \emcite{miikkulainen:psylm97}.  The TAE may
be merely a minor consequence of this multi-level architecture for
representing correlations over a wide range of time scales.

A main contribution of the RF-LISSOM model of the TAE is its novel
explanation of the indirect effect.  Proponents of the lateral
inhibitory theory of direct effects have generally ignored indirect
effects, or postulated that they occur only at higher cortical levels 
\cite{wenderoth:vres88}, partly because it has not been clear how they
could arise through inhibition in V1.  
%%
RF-LISSOM demonstrate that a quite simple, local 
mechanism in V1 is sufficient to produce indirect effects.  If the
total synaptic resources at each neuron are limited, strengthening the
lateral inhibitory connections between active neurons weakens their
inactive inhibitory connections.  There is widespread biological
evidence of competition for a limited number of synaptic sites
\cite*{bourgeois:synaptogenesis,hayes:optic,murray:target,pallas:compensation,purves:neurotrophic}.
There is also extensive computational justification for synaptic
resource conservation, beginning with one of the first computational
models of Hebbian adaptation \cite*{rochester:iretit56}.  Without
such normalization, connection weights governed by a Hebbian rule will
increase indefinitely, or else each would reach a maximum strength
\cite{miller:rolenc}.  Neither outcome would appear biologically or
computationally plausible, so the assumption of some form of
normalization is well-motivated \cite{sirosh:phd}.

Through mechanisms similar to those causing the TAE, the RF-LISSOM
model should also be able to explain simultaneous tilt illusions
between spatially separated stimuli.  Such an explanation was
originally proposed by \emcite{carpenter:interactions}.  However, it
will be necessary to train the system with inputs that have
longer-range correlations between similar orientations, such as
sinusoidal gratings (representing objects with parallel lines).  With
such patterns, long-range connections develop between widely separated
orientation detectors, in addition to the relatively local connections
now present.  Trained with such patterns, RF-LISSOM should be able to
account for tilt illusions as well as tilt aftereffects.  Although
such experiments require even larger cortex and retina sizes, they
should become practical in the near future.

In addition, many similar phenomena such as aftereffects of curvature,
motion, spatial frequency, size, position, and color have been
documented in humans~\cite{barlow:aftereffects}. Since specific
detectors for most of these features have been found in the cortex,
RF-LISSOM should be able to account for them by the same process of
decorrelation mediated by self-organizing lateral connections.

%% This document doesn't happen to have a table, but here is an
%% example of one from a different document.  Just uncomment it and
%% recompile to see the output.  See the sample talk for examples of 
%% putting two floating things, like figures or tables, side by side
%% using minipage.
%%
%\begin{table}
%  \small
%  \tabcolsep 4pt
%  \renewcommand{\arraystretch}{0.7}
%  \centering
%  \begin{tabular}{|ll|}
%    \hline
%    \textbf{Feature}        & Set to 1 for words\rule[-4pt]{0pt}{14pt}\\
%    \hline
%    \hline
%    \textbf{1. Ball}        & \texttt{ball}, \texttt{baseball} and \texttt{dance}\rule[-2pt]{0pt}{12pt}\\
%    \textbf{2. Verb}        & \texttt{thrown}, \texttt{tossed} and \texttt{hosted} \\
%    \textbf{3. Other}       & \texttt{the} and \texttt{was} \\
%    \textbf{4. Preposition} & \texttt{in}, \texttt{for}, and \texttt{by} \\
%    \textbf{5. Location}    & the five \emph{location} words and \texttt{in} \\
%    \textbf{6. Recipient}   & the five \emph{recipient} words and \texttt{for} \\
%    \textbf{7. Agent}       & the five \emph{agent} words and \texttt{by}\rule[-4pt]{0pt}{10pt}\\
%    \hline
%    \textbf{8. Sense}       & Graduated according to word sense\rule[-4pt]{0pt}{14pt}\\
%    \hline
%  \end{tabular}
%  \vspace{-0.05in}% Make figure a little tighter
%  \titledcaption{tb:features}{The word representation vectors}{
%    The words in the lexicon were encoded by these eight features.
%    The first seven components were set to either 0 or 1; the right column
%    lists those words that had the value 1.  The \textbf{Sense} feature was
%    used to indicate the degree of association to which a word had the two
%    senses of \texttt{throw} and \texttt{ball}.}
%\end{table}


%%%%%%%%%%%%%%%%%%%%%%%%%%%%%%%%%%%%%%%%%%%%%%%%%%%%%%%%%%%%%%%%%%%%%%%%%%%%%%%
\section{Conclusion}

The experiments reported in this paper lend strong computational
support to the theory that tilt aftereffects result from Hebbian
adaptation of the lateral connections between neurons.
Furthermore, the aftereffects occur as a result of the same
decorrelating process that is responsible for the initial development
of the orientation map.  This process tends to deemphasize constant
features of the input, resulting in short-term perceptual anomalies
such as aftereffects.  The same model should also apply to other 
aftereffects and to simultaneous tilt illusions.

Because RF-LISSOM is a computational model, it can demonstrate many
phenomena in high detail that are difficult to measure experimentally,
thus presenting a view of the cortex that is otherwise not available.
This type of analysis can provide an essential complement to
experimental work with humans and animals.
%%
RF-LISSOM provides a comprehensive and fundamental account of how both
cortical structure and function emerge by Hebbian self-organization in
the primary visual cortex.  It also shows how both indirect and direct
tilt aftereffects could arise from simple, biologically plausible
mechanisms in the primary visual cortex.  Thus a single simple
computational model may lead to significant insights into a variety of
cortical phenomena, and thereby contribute to our understanding of the
cortex.

%%%%%%%%%%%%%%%%%%%%%%%%%%%%%%%%%%%%%%%%%%%%%%%%%%%%%%%%%%%%%%%%%%%%%%%%%%%%%%%
\appendix 
\section{Acknowledgments}

Thanks to Joseph Sirosh for supplying the RF-LISSOM code. This
research was supported in part by the National Science Foundation 
under grant \#IRI-9309273.  Computer time for the simulations was
provided by the Pittsburgh Supercomputing Center under grant
IRI940004P.

%%%%%%%%%%%%%%%%%%%%%%%%%%%%%%%%%%%%%%%%%%%%%%%%%%%%%%%%%%%%%%%%%%%%%%%%%%%%%%%
%% Change the title of \thebibliography section if required
%%\renewcommand\refname{Bibliography}
%%
%% Squeeze bibliography together
%%\renewcommand{\baselinestretch}{0.7}
\bibliographystyle{nnapalike}
%%\setlength{\bibhang}{0.125in}  % To match CogSci97 specs
\bibliography{nnstrings,nn}

\end{document} 
