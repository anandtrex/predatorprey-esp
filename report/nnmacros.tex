%% $Id: nnmacros.tex,v 1.28 2005/09/28 21:18:37 jbednar Exp $
%%
%% Standard set of LaTeX macros and declarations for the NN group
%%
%% Originally based on risto's com.tex
%%
%% Because of \usepackage{hyperref}, this file should be included
%% *after* all other packages.


%%%%%%%%%%%%%%%%%%%%%%%%%%%%%%%%%%%%%%%%%%%%%%%%%%%%%%%%%%%%%%%%%%%%%%%%%%%%%%%
%% Add ability to check for PDFTeX

\newif\ifpdf
\ifx\pdfoutput\undefined
  \pdffalse
\else
  \pdfoutput=1
  \pdftrue
\fi

%% Macro useful for figuring out if a TeX command has been defined
%%
\def\ifundefined#1{\expandafter\ifx\csname#1\endcsname\relax}

%%%%%%%%%%%%%%%%%%%%%%%%%%%%%%%%%%%%%%%%%%%%%%%%%%%%%%%%%%%%%%%%%%%%%%%%%%%%%%%
%% Packages required


% Because of this call, the "graphics" interface can't be used in
% conjunction with this file.  
%
\ifpdf\PassOptionsToPackage{pdftex}{graphicx}\fi
\usepackage{graphicx}

%% If you want to supply options to the hyperref package, you can add :
%% 
%% \PassOptionsToPackage{...}{hyperref} 
%%
%% in your source file before any \usepackage commands.  To turn
%% off hyperlinks altogether, you can use: 
%%
%% \PassOptionsToPackage{draft}{hyperref} %% Turn off hyperref links
%%
\ifpdf
%% Highlights links in PDF files
\usepackage{hyperref}
%% Highlights bibliography links
\newcommand{\biblink}[2]{\htmladdnormallink{#1}{#2}}
\fi

% From the LaTeX2HTML distribution.
%\usepackage{html}%   
%
% If you don't have or want LaTeX2HTML when running this at home, just
% copy /u/nn/tex/latex2html/texinputs/html.sty to somewhere   
% in your TEXINPUTS path and ignore the rest of LaTeX2HTML.


%%%%%%%%%%%%%%%%%%%%%%%%%%%%%%%%%%%%%%%%%%%%%%%%%%%%%%%%%%%%%%%%%%%%%%%%%%%%%%%
%% From risto's com.tex:

%%% Generates a \tt quotation without indentation
%\def\quotation{\list{}{\listparindent 0.0em
%    \setlength{\parindent}{0.0in}
%    \renewcommand{\baselinestretch}{1}\small\normalsize\tt
%    \renewcommand{\arraystretch}{1}
%%    \itemindent\listparindent
%    \setlength{\leftmargin}{0.0in}
%    \parsep 0pt plus 1pt}\item[]}
%\let\endquotation=\endlist

%% I like to tighten up my captions and tables
\newcommand{\mycapdefs}{\renewcommand{\baselinestretch}{1}\small}
\newcommand{\mytabdefs}{\renewcommand{\arraystretch}{1}\small}

%% Get it?
\hyphenation{Miik-ku-lai-nen}

%% I don't like pages with only figures on them
\renewcommand{\topfraction}{0.9}
\renewcommand{\floatpagefraction}{0.8}
\renewcommand{\dbltopfraction}{0.9}
\renewcommand{\dblfloatpagefraction}{0.8}
\renewcommand{\textfraction}{.15}

%% Simple PostScript picture inclusion macros using the graphicx package
%%
%% These three commands have been provided for backwards compatibility
%% only.  You should not use them in new documents, because the 
%% non-standard argument format causes trouble with packages that expect
%% LaTeX-style  arguments, such as AucTeX, emacs font-lock, and
%% LaTeX2HTML.  Instead use the \pspic or \pdfpic macros (below).
%%
\begin{latexonly}
\def\putpicture   #1 (#2)\centerline{\includegraphics[totalheight=#2]ps/#1.ps}
\def\hputpicture  #1 (#2){\centerline{\includegraphics[width=#2]{ps/#1.ps}}}
\def\hvputpicture #1 (#2 by #3){\centerline{\includegraphics[width=#2,totalheight=#3]{ps/#1.ps}}}
\end{latexonly}


%%%%%%%%%%%%%%%%%%%%%%%%%%%%%%%%%%%%%%%%%%%%%%%%%%%%%%%%%%%%%%%%%%%%%%%%%%%%%%%
%% \scribbledraft: Provides a single-column mode with lots of space
%% to write comments on, e.g. for giving to a coauthor.
%%
%% To use, add \scribbledraft before \begin{document} but after
%% all your \usepackage commands.
%%
%% Tips:
%%
%% If the formatting doesn't look right, you may have to comment out
%% your document style (e.g. \usepackage{<confname>}) and instead use
%% \usepackage{nntwocolumn}.  Usually it will only be the first 
%% page that has problems, though; that's where style files usually
%% mess with the page widths and lengths.
%%
%% If you get lines sticking past the edge of the column, try adding
%% \sloppy just after \scribbledraft.
%%
%% If your document has figure or figure* environments that come out much
%% too small in \scribbledraft mode, you may be able to fix them by
%% changing them to fullfigure or fullfigure* environments, which are
%% defined below.  However, you may find that the result is no longer
%% centered properly or is the wrong width.  If so, you might need to add
%% a \setlength\originalwidth{6.5in} or \setlength\marginoffset{0in}
%% after \scribbledraft, adjusting those two values to fix the width and
%% left margin.  
%%
\newcommand{\scribbledraft}{
  \newlength\marginoffset\setlength\marginoffset{-0.5in}
  \newlength\nominalwidth\setlength\nominalwidth{3.5in}
  \newlength\nominalmargin\setlength\nominalmargin{0.5in}
  \setlength\topmargin{0.5in}
%% \setlength\oddsidemargin{0.5in}
%% \setlength\evensidemargin{0.5in}
\setlength\oddsidemargin{\nominalmargin}
\setlength\evensidemargin{\nominalmargin}
\setlength\textheight{8in}
\setlength\textwidth{\nominalwidth}
\setlength\footskip{20pt}
\setlength\marginparwidth{2.5in}
\pagestyle{plain}
\onecolumn
% Some styles (e.g. cogsci, nntwocolumn) call \twocolumn in \maketitle,
% so here we define it as a no-op
\renewcommand{\twocolumn}[1][]{##1\bigskip\bigskip} 
%%
\renewcommand{\forcedoublecolumnfigurestart}
{\setlength\textwidth{\originalwidth}%
\hspace*{\marginoffset}%
\begin{minipage}{\originalwidth}}
%%
\renewcommand{\forcedoublecolumnfigureend}
{\end{minipage}\setlength\textwidth{\nominalwidth}}
}
%% Save original value here for later
\newlength{\originalwidth}\setlength{\originalwidth}{\textwidth}
%%
\newcommand{\forcedoublecolumnfigurestart}{}
\newcommand{\forcedoublecolumnfigureend}{}
%%
\newenvironment{fullfigure}[1][tbp]
{\begin{figure}[#1]\forcedoublecolumnfigurestart}
{\forcedoublecolumnfigureend\end{figure}}
%%
\newenvironment{fullfigure*}[1][tbp]
{\begin{figure*}[#1]\forcedoublecolumnfigurestart}
{\forcedoublecolumnfigureend\end{figure*}}



%%%%%%%%%%%%%%%%%%%%%%%%%%%%%%%%%%%%%%%%%%%%%%%%%%%%%%%%%%%%%%%%%%%%%%%%%%%%%%%
%% Conditional picture inclusion macros using the graphicx package
%%
%% If you use \pdfpic (below) when specifying a picture, you
%% can use \psdraftlevel (also below) to specify which ones to include
%% for a printout (e.g. to save print time or file size, or to print
%% color pages separately).  It's also handy to have the type of picture
%% declared in the source file this way, as a clue to what type of
%% picture is involved.  Just ignore this if it is not needed.

%% First, declare some symbolic constants representing the different
%% classes of images.  These definitions shouldn't ever need changing, 
%% but new ones can be added in between if needed.
%%  
\def\psnone{100}        % No picture should be included
\def\psline{80}         % Black and white line drawings only
\def\psmono{60}         % Monochrome images only
\def\pscolororbw{40}    % Images for which color is optional
\def\pscolor{20}        % Images for which color is required
\def\psall{0}           % All pictures should be included


%% The \psdraftlevel command determines which pictures should actually
%% be included, while other pictures merely have their bounding
%% box drawn in to save time or file size, or to print on printers
%% with insufficient memory.  It is optional, but if used, should
%% presumably be in the preamble. Each of the above levels is
%% inclusive of the preceding ones (those with lower numbers), so
%% \mono includes line drawings too, and \color includes line, mono,
%% and color pictures. 
%%
%% Example:  To include all pictures (the default),
%%   \psdrafttype{\psall}
%%
%% Example:  To include all pictures which make sense in monochrome,
%%   \psdrafttype{\pscolororbw}
%%
%% Example:  To include only line drawings, i.e. pictures likely to be
%% very small in file size, 
%%   \psdrafttype{\psline}
%%
%% Below, 0 is hardcoded instead of \psall because of problems 
%% with some versions of latex2html
\newcount\psdraftlevel \psdraftlevel=0
\newcommand{\psdrafttype}[1]{\psdraftlevel=#1}


%% Including PDF, JPG, PNG, or TIF pictures in PDFLaTeX documents,
%% (as well as EPS pictures in LaTeX documents)
%%
%% \pdfpic[category]{width}{height}{[dir/]filename}{filetype}
%%
%%
%% Arguments:
%%
%% The "category" argument is optional; if present it should indicate the
%% type of image the file represents, i.e. \psline for a line drawing,
%% \pscolor for a color photo, \pscolororbw for an image that is readable
%% either in color or monochrome, \psnone for a blank space, etc.  For 
%% a given run, the actual image is included only if the the
%% \psdrafttype is low enough.  For example, if \psdrafttype}{\psmono}
%% has been specified, all line and monochrome pictures are included,
%% but color pictures will be represented only by a bounding box.  If
%% "category" is omitted then the image will always be displayed,
%% regardless of draftlevel.
%%
%% Either or both of the length arguments "length" and "width" can be 
%% omitted if they are replaced by "!", which indicates that that
%% dimension should scale proportionally.  (If both are "!", then the
%% image is left at the natural size of the image, whatever that might
%% be for a given filetype.)  A dummy image (type \psnone) must have
%% both "length" and "width" specified, since they cannot be read from
%% a file.
%%
%% The argument "[dir/]filename" should be the complete path to your
%% picture file, minus the final extension, which in turn goes in the 
%% {filetype} argument.  For instance, to access "./images/pretty.jpg",
%% use \pdfpic{!}{!}{images/pretty}{jpg}.  
%%
%% 
%% Tips and examples:
%% 
%% Filetypes (i.e., extensions) of jpg, png, pdf and tif are currently
%% supported. Of these you should usually use pdf for line drawings,
%% jpg for photographs, and tif or png for non-photograph bitmap images.
%% 
%% Usually, it is best to specify that a picture be fixed in size
%% horizontally if it extends across the page, so that it still fits
%% on the page properly if the margins, number of columns, etc. are
%% changed.  
%%
%% Example: To insert a monochrome PNG picture images/sample.png, scaled
%% to be as wide as the current page,        
%%   \pdfpic[\psmono]{\textwidth}{!}{images/sample}{png}
%%
%% Example: To insert a line drawing images/sample.pdf, scaled to be as
%% half as wide as the page, rotated 90 degrees counterclockwise, and
%% framed with a thin, snug outline,
%%   \frame{\rotatebox{90}{\pdfpic[\psline]{!}{0.5\textwidth}{images/sample.pdf}{pdf}}}
%%
%% Example: To insert a dummy 3" x 2" space for pasting in a picture 
%% later, when no image file actually exists on disk,
%%   \pdfpic[\psnone]{3in}{2in}{Paste feller:science96 figure 3 here}{}
%%
%% If this command is used under LaTeX, the filetype is ignored and 
%% one of type eps and ending in .ps is assumed.  If you do use such EPS 
%% files, make sure that they are truly EPS (i.e., have a bounding box)
%% and not just PostScript.
%%
%%
\newcommand{\pdfpic}[5][\psall]{\includepdfpic[#1]{#2}{#3}{#4}{#5}{}}%


%% Same as \pdfpic but clips to boundary of box
\newcommand{\pdfpicclip}[5][\psall]{\includepdfpic[#1]{#2}{#3}{#4}{#5}{clip=True,}}%



%% Old-style interface for inserting PostScript pictures into LaTeX
%% documents:
%%
%%
%% This interface only supports PDF and EPS filetypes, and is included
%% for backwards compatibility with older documents developed in LaTeX.
%% Such documents can be converted to work with PDFLaTeX simply by
%% converting all of the EPS files they include to PDF.  For instance,
%% (in csh):
%%
%% foreach psfile (ps/*.ps) 
%%   epstopdf "${psfile}"
%% end
%%
%% That way PDFLaTeX will find the ps/*.pdf files, and LaTeX will find
%% the ps/*.ps files.  Be sure to check the generated .pdf figures
%% carefully; sometimes the translation process generates visible
%% artifacts.  For instance, sometimes you can see fringes around
%% sharp, high-contrast borders in images, like those when converting
%% bitmaps to JPEG.
%%
%% 
%% This macro is used just like \pdfpic, except that you should omit the
%% filetype (because only one filetype is supported for each platform)
%% and the directory (which is assumed to be ./ps/):
%%
%% \pspic[category]{width}{height}{filename}
%%
\newcommand{\pspic}[4][\psall]{\pdfpic[#1]{#2}{#3}{ps/#4}{pdf}}
\newcommand{\pspicclip}[4][\psall]{\pdfpicclip[#1]{#2}{#3}{ps/#4}{pdf}}




%%%%%%%%%%%%%%%%%%%%%%%%%%%%%%%%%%%%%%%%%%%%%%%%%%%%%%%%%%%%%%%%%%%%%%%%%%%%%%%
%%
%% Implementation for \pdfpic, etc.; shouldn't usually 
%% be called directly.  Arguments are same as for \pdfpic
%% except that a 6th argument is accepted, and passed
%% directly to \includegraphics
%%
\newcommand{\includepdfpic}[6][\psall]{%
\def\pdfpicFileType{#5}%
\def\pdfpicFileExt{#5}%
%%
%% Assume .ps under LaTeX
\ifpdf\else% 
\def\pdfpicFileType{eps}%
\def\pdfpicFileExt{ps}%
\fi%
%%
%% Dummy picture: empty box (not even a frame).
%% Unspecified heights default to \unitlength
\ifx#1\psnone%
\resizebox{#2}{#3}{\begin{picture}(1,1)\end{picture}}%
%%
%% Draft picture: framed box with filename inside.
%% This looks fine with dvips, but it confuses xdvi
%% for some reason, which draws an extra box with the
%% natural size of the image. 
\else{\ifnum#1<\psdraftlevel%
\frame{\resizebox{#2}{#3}{\includegraphics[#6draft=true,type=\pdfpicFileType,ext=.\pdfpicFileExt,read=*,command=##1]{#4}}}%
\else%
%%
%% Normal picture: image in the specified format.
\resizebox{#2}{#3}{\includegraphics[#6type=\pdfpicFileType,ext=.\pdfpicFileExt,read=*,command=##1]{#4}}\fi}\fi%
}

%% Simpler version to keep LaTeX2HTML from getting confused
\begin{imagesonly}
\renewcommand{\pspic}[4][\psall]{\resizebox{#2}{#3}{\includegraphics{ps/#4.ps}}}
\end{imagesonly}



%%%%%%%%%%%%%%%%%%%%%%%%%%%%%%%%%%%%%%%%%%%%%%%%%%%%%%%%%%%%%%%%%%%%%%%%%%%%%%%
%%>> To do (6/2002): 
%% It would be nice to have some sort of facility for automatically 
%% handling conditional text for different document types.  Here's
%% a sketch, but it is not legal LaTeX code:
%%
%% \newcommand{\definedocumenttype}[2]{
%%   \expandafter\def\csname dt#1\endcsname{#2}%
%%   \newcommand{not#1}[1]{}%
%%   \newcommand{#1only}[1]{}%
%% }
%% \definedocumenttype{book}{100}          % Book
%% \definedocumenttype{thesis}{80}         % PhD. or Master's thesis
%% \definedocumenttype{techreport}{60}     % Technical report
%% \definedocumenttype{preprint}{45}       % Journal preprint (to be passed around, not submitted)
%% \definedocumenttype{journal}{40}        % Journal paper (as published, or to be submitted)
%% \definedocumenttype{conf}{20}           % Conference paper
%% \definedocumenttype{article}{0}         % Default; generic article
%% 
%% \newcount\documenttype \documenttype=0
%% \newcommand{\setdocumenttype}[1]{%
%%   \documenttype=#1%
%%   \newcommand{not#1}[1]{#1}%
%%   \newcommand{#1only}[1]{}%
%% }
%%
%% Usage:
%%  \setdocumenttype{\book}  % Or any other
%%  \bookonly{Text for the book version}
%%  \notbook{Text for all versions except the book}
%%
%% All 'only' and 'not' comparisons should be available
%% even if only a single document type is available.  It should
%% be legal to define multiple document types.  However, it's not
%% clear how to implement any of this.


%%%%%%%%%%%%%%%%%%%%%%%%%%%%%%%%%%%%%%%%%%%%%%%%%%%%%%%%%%%%%%%%%%%%%%%%%%%%%%%
%%  Standard lengths, used to keep documents uniform throughout
 
%% Set size of windowpane to fit one, two, etc. square panes per line  
%%
\newlength{\oneplotwidth}           \setlength{\oneplotwidth}         {\textwidth}
\newlength{\twoplotwidth}           \setlength{\twoplotwidth}         {0.46\textwidth}
\newlength{\twoplotheight}          \setlength{\twoplotheight}        {\twoplotwidth}
\newlength{\threeplotwidth}         \setlength{\threeplotwidth}       {0.31\textwidth}
\newlength{\threeplotheight}        \setlength{\threeplotheight}      {\threeplotwidth}
\newlength{\fourplotwidth}          \setlength{\fourplotwidth}        {0.22\textwidth}
\newlength{\fourplotheight}         \setlength{\fourplotheight}       {\fourplotwidth}
\newlength{\fiveplotwidth}          \setlength{\fiveplotwidth}        {0.18\textwidth}
\newlength{\fiveplotheight}         \setlength{\fiveplotheight}       {\fiveplotwidth}
 
%% Set size of windowpane to fit the specified configuration of square
%% panes per page  
%%
\newlength{\twobytwoplotwidth}      \setlength{\twobytwoplotwidth}    {0.46\textwidth}
\newlength{\threebyfourplotwidth}   \setlength{\threebyfourplotwidth} {0.25\textwidth}
\newlength{\fourbythreeplotwidth}   \setlength{\fourbythreeplotwidth} {0.171875\textheight}
%% Note that the portrait page above with 12 square panes is named as 
%% a "width", but it is actually limited in the vertical direction
%% for typical US paper proportions, assuming there is a 1-inch caption.
 
 
%% Plot width that xmgr graphs should use
%%
%% This value is appropriate for output that has been run through
%% ps2epsi to fix the bounding boxes.  Xmgr computes them too
%% generously, so this would need to be some smaller value otherwise.
%%
\newlength{\xmgrplotwidth}          \setlength{\xmgrplotwidth}{\textwidth}
 
 
%% Blank variable available for various purposes
%%   Must use \setlength or \settowidth to give it 
%%   a value before using it
%%   E.g.: \settowidth{\templength}{Some text}
%%
\newlength{\templength}

 

%%%%%%%%%%%%%%%%%%%%%%%%%%%%%%%%%%%%%%%%%%%%%%%%%%%%%%%%%%%%%%%%%%%%%%%%%%%%%%%
%% Support for captioned subfigures
%%
%% To enable these commands, add "\usepackage[tight,center]{subfigure}"
%% before loading this file.  Then define subfigures with \sfobject, \sfpic,
%% or \sfpicframe, and refer to them with \sfref or \sfrefnp.
%%
%% Requires subfigure package version 2.1.2 or higher.  See the subfigure 
%% documentation for more details, such as how to change the caption 
%% separation, fonts, etc.:
%% http://www-2.cs.cmu.edu/afs/cs/usr/sdc/www/latex/subfigure.html
%%
\makeatletter
  \@ifpackageloaded{subfigure}{%

  %% Set up the subfigure package to italicize only the subfigure letters,
  %% not the parentheses, and to make \ref show only the bare letters,
  %% without any parentheses and without the main figure number.
  \let\p@subfigure\relax
  \renewcommand{\thesubfigure}{\emph{\alph{subfigure}}}
  \renewcommand*{\@thesubfigure}{(\thesubfigure)\hskip\subfiglabelskip}
  \renewcommand*{\@@thesubfigure}{(\thesubfigure)}


  %% Define a subfigure of any type with a caption underneath it
  %% Usage:   \sfobject[width]{item}{caption-text}{subfigure-label}
  %% Example: \sfobject{0.3\textwidth}{... a picture of some sort ...}{Above is a fancy picture}{fig}
  %% The subfigure-label must be unique for the whole document.
  %%
  \newcommand{\sfobject}[4]{%
    \subfigure[\label{sf:#4}#3]{\begin{minipage}[t]{#1}#2\end{minipage}}}
  \newcommand{\sfobjectuncaptioned}[2]{% Same spacing but no caption
    \subfigure{\begin{minipage}[t]{#1}#2\end{minipage}\addtocounter{subfigure}{-1}}}


  %% Define a subfigure consisting of a captioned \pdfpic
  %% Usage:   \sfpic{width}{filenamebase}{imagetype}{caption text}{subfigure-label}
  %% Example: \sfpic{0.3\textwidth}{picture}{jpg}{A picture}{renoir-pic}
  %%
  \newcommand{\sfpic}[5]{%
    \sfobject{#1}{\pdfpic{\textwidth}{!}{#2}{#3}}{#4}{#5}}%
  \newcommand{\sfpicuncaptioned}[3]{% Same but no caption
    \sfobjectuncaptioned{#1}{\pdfpic{\textwidth}{!}{#2}{#3}}}%

  %% Same as \sfpic but with a frame around the picture
  \newcommand{\sfpicframe}[5]{%
    \sfobject{#1}{\frame{\pdfpic{\textwidth}{!}{#2}{#3}}}{#4}{#5}}%
  \newcommand{\sfpicframeuncaptioned}[3]{%
    \sfobjectuncaptioned{#1}{\frame{\pdfpic{\textwidth}{!}{#2}{#3}}}}%


  %% Expand a reference to a subfigure, e.g. in caption or the main text
  %% \sfref{subfigure-label} yields (\emph{a}), etc.
  \newcommand{\sfref}[1]{(\ref{sf:#1})}


  %% Same as \sfref without the parentheses
  %% \sfrefnp{subfigure-label} yields \emph{a}, etc.
  \newcommand{\sfrefnp}[1]{\ref{sf:#1}}


  %% If producing overhead slides, call this command in the preamble
  %% so that (a), (b), etc. will be omitted from subfigure captions
  \newcommand{\sfstriplabels}{\renewcommand*{\@thesubfigure}{\mbox{}}}
}
\makeatother



%%%%%%%%%%%%%%%%%%%%%%%%%%%%%%%%%%%%%%%%%%%%%%%%%%%%%%%%%%%%%%%%%%%%%%%%%%%%%%%
%%  Misc

%% Puts a tiny note in the margin; useful for notes on drafts.
%%
%% To disable all of them uncomment the second def.
%% When you use it in the middle of a paragraph, you should follow 
%% the last brace with a comment character (%) to avoid getting
%% a double space in the output.
\newcommand{\draftnote}[1]{{\marginpar{\tiny\em\raggedright #1}}}
%%\renewcommand{\draftnote}[1]{}

%% The \draftnote command is based on \marginpar, which doesn't work 
%% in floating  bodies like figures and tables because the notes also
%% float.  So we provide a non-floating version of marginpar
%% (\rawmarginpar), and then use that to create a version of
%% \draftnote that works with figures (\rawdraftnote).  Note that
%% \rawdraftnotes might overlap, go off the page, etc.; they are not
%% juggled automatically like \draftnotes are.  \rawdraftnote
%% usually works best right before the figure caption.
%%
\newcommand{\rawmarginpar}[1]{%
\vadjust{\smash{\llap{\parbox[t]{0.7\marginparwidth}{#1}\kern\marginparsep}}}}
\newcommand{\rawdraftnote}[1]{{\rawmarginpar{\tiny\em\raggedright #1}}}

%% Pretends that its argument takes zero horizontal space, e.g. to 
%% prevent a caption from having a linebreak by letting it 
%% extend slightly into the space on either side.
%%
\providecommand{\zerospace}[1]{\mbox{\makebox[0in]{#1}}}

%% Same as \zerospace except that the result is left-
%% justified, as if it had zero length out to the right.
\providecommand{\zerolength}[1]{\mbox{\makebox[0in][l]{#1}}}


%% Centers the given object vertically and horizontally
%% within a square box of the specified size, ignoring 
%% the size of the object.  
%%
%% Usage: \squarebox[position]{width}{object}
%%
%% Examples:
%%   Center some text in the left 1/10 page and draw a border around it:
%%     \frame{\squarebox{0.1\textwidth}{Some text}}
%%   Center an image 1/8 the size of the page in a box twice its width:
%%     \frame{\squarebox{0.25\textwidth}{\pdfpic{0.5\textwidth}{!}{somefile}{pdf}}
\newcommand{\squarebox}[3][c]{%
    \begin{minipage}[#1]{#2}%
      \setlength{\unitlength}{\textwidth}%
      \begin{picture}(1.0,1.0)%
        \put(0.5,0.5){\makebox(0,0)[c]{#3}}%
      \end{picture}%
    \end{minipage}}

%% Command to add a degree symbol following a number
%%
%% When using it, you need to follow it with a "\ " if the next
%% character should be a space.  
%%
%% Examples: "23\degree\ line"    "the orientation was 23\degree."
%%
%% Used to be called \deg but there is now a LaTeX2e command by that
%% name. 
\newcommand{\degree}{\ensuremath{^\circ}}


%% Non-math-mode equivalents of math-mode characters, etc.
%% These are useful for preventing LaTeX2HTML from converting
%% tiny in-text expressions like these into GIFs, and can also 
%% make text a little more readable overall.
\newcommand{\superscript}[1]{\ensuremath{^#1}}
\newcommand{\subscript}[1]{\ensuremath{_#1}}
\newcommand{\mtimes}[2]{\ensuremath{#1 \times #2}}
\newcommand{\madd}[2]{\ensuremath{#1 + #2}}
\newcommand{\mplus}{\ensuremath{+}}
\newcommand{\mpm}{\ensuremath{\pm}}
\newcommand{\mlt}{\ensuremath{<}}
\newcommand{\mgt}{\ensuremath{>}}

%% Given a color type as argument, e.g. \psmono or \pscolor, returns the note 
%% that should be placed in the caption for that figure.
%% This will make figures that require color say so; I can't think of anything
%% useful for the rest to say.
%% Maybe \pscolororbw should say something like "(Figure looks better in color)"?
\newcommand{\figurenote}[1]{%
\ifx#1\pscolor{\ ({\em color figure\/})}\fi}


%% Dummy command marking a section of text as something for latex2html
%% to typeset as a single left-justified block  
\newcommand{\htmlblock}[1]{#1}


%% Constructs a titled figure caption.  Takes four arguments:
%%
%% \titledcaption[category]{label}{title}{captiontext}
%%
%% The "category" argument is optional; if present it should indicate the
%% type of image the file represents, i.e.  \psline for a line drawing,
%% \pscolor for a color photo, \pscolororbw for an image that is readable
%% either in color or monochrome, \psnone for a blank space, etc.  At
%% present, "category" is only used to see whether or not to
%% add a descriptive note for the title and list of figures entry.
%% See the \figurenote command for the cases where a note will be
%% added.  The default will never add any note.
%% 
%% The "label" is a name used to construct a label for the figure.
%%
%% The "title" should be a short heading, which will go in bold at the
%% top of the caption, and also goes into the list of figures.
%%
%% The "captiontext" will go into the caption only, not the list of
%% figures, after the title.  Just use an empty last argument {} if 
%% there is no caption text.
%% 
%% Complete figure example:
%% \begin{figure}
%%   \centering 
%%   \pspic[\psline]{\oneplotwidth}{!}{some-figure-filename-wo-ps-extension}
%%   \titledcaption[\psline]{fg:some-figure-label}
%%   {Title of the figure}{ 
%%     Some long text describing what this figure means and how it
%%     should be interpreted.}
%% \end{figure}
%%
%% jbednar040510: For some reason, I had replaced both "#1"s with 
%% "\protect{#1}", probably to fix some problem with the separate
%% versions for typesetters (below).  However, that broke the color
%% figure labels for the normal case, so I have removed it.  Next time
%% the typesetter versions are used, they will probably need to be
%% debugged.
%%
\newcommand{\titledcaption}[4][\psall]{%
    \caption[#3\figurenote{#1}]{%
      \label{#2}\mycapdefs%
      \textbf{\mathversion{bold}#3\figurenote{#1}.} %
      \htmlblock{#4}}}


%%%%%%%%%%%%%%%%%%%%%%%%%%%%%%%%%%%%%%%%%%%%%%%%%%%%%%%%%%%%%%%%%%%%%%%%%%%%%%%
%% Support for defining figure graphics separately from captions
%%
%% To use this, just pick a label you want to refer to the figure by,
%% and define the meat of the figure:
%%
%% \definefigbody{fg:some-fig}{%
%%  \sfpicframe{\twoplotwidth}{picture1}{jpg}{Case 1}{case-1-label}\hfill
%%  \sfpicframe{\twoplotwidth}{picture2}{jpg}{Case 2}{case-2-label}}
%%
%% Then wherever you want the figure to appear, insert it with a caption:
%%
%%  \insertfig[\pscolor]{fg:some-fig}{%
%%    Figure title}{%
%%    This figure conclusively demonstrates that...}%
%%
%% Leave off [\pscolor] if you don't want the "(color figure)" note
%% in the caption.
%%
%% Everything else will be handled automatically.  If you want fine
%% control over the figure (instead of automatically using
%% titledcaption, fullfigure*, etc.), just use 
%% \insertfigbody{fg:some-fig} to insert the meat of the figure into 
%% your own figure skeleton.
%%
%% If you also want to define the captions separately from where they
%% are first inserted into the text, then use \definefigcaption
%% instead of \insertfig.  That way the caption will be defined but
%% not yet inserted into the text.  Then just use \firstfigref
%% instead of \ref the first time you refer to the figure, and the
%% figure will automatically be placed into the text at that point.
%%
%% If you want the captioned figure to be placed somewhere other
%% than the first reference, just put an \insertfighere{fg-some-fig} 
%% command where you want the figure to appear.
%%
%%
%% Defines the body of the given figure
\newcommand{\definefigbody}[2]{%
  \ifundefined{figbody--#1}\expandafter\def\csname figbody--#1\endcsname{#2}%
  \else\errmessage{Figure #1 already defined}\fi}%
%%
%% Rarely used by end users:
\newcommand{\insertfigbody}[1]{\csname figbody--#1\endcsname}%
\newcommand{\insertfigtype}[1]{\csname figtype--#1\endcsname}%
\newcommand{\insertfigtitle}[1]{\csname figtitle--#1\endcsname}%
\newcommand{\insertfigcaption}[1]{\csname figcaption--#1\endcsname}%
%%
%% Defines the type, title, and caption of the given figure
\newcommand{\definefigcaption}[4][\psmono]{%
  \ifundefined{figtype--#2}\expandafter\def\csname figtype--#2\endcsname{#1}%
  \else\errmessage{Type for #2 already defined}\fi%
  \ifundefined{figtitle--#2}\expandafter\def\csname figtitle--#2\endcsname{#3}%
  \else\errmessage{Title for #2 already defined}\fi%
  \ifundefined{figcaption--#2}\expandafter\def\csname figcaption--#2\endcsname{#4}%
  \else\errmessage{Caption for #2 already defined}\fi}%
%%
%% Inserts the given figure given the type, title, and caption
\newcommand{\insertfig}[4][\psmono]{%
  \begin{fullfigure*}\insertfigbody{#2}\titledcaption[#1]{#2}{#3}{#4}\end{fullfigure*}}%
%%
%% Rarely used by end users:
\newcommand{\insertfighere}[1]{%
  \insertfig%
  [\insertfigtype{#1}]{#1}%
  {\insertfigtitle{#1}}%
  {\insertfigcaption{#1}}}%
%%
%% Refers to and inserts the given captioned figure
\newcommand{\firstfigref}[1]{\ref{#1}\insertfighere{#1}}%


%%%%%%%%%%%%%%%%%%%%%%%%%%%%%%%%%%%%%%%%%%%%%%%%%%%%%%%%%%%%%%%%%%%%%%%%%%%%%%%
%% Support for figures at end of document for journal paper
%% submissions
%% 
%% None of these commands are used in a normal camera-ready paper
%%
%% For these commands to work, you should use \firstfigref
%% consistently (which allows the actual caption to be inserted 
%% wherever).  
%%
%% If you use subfigures, you should also add a line:
%%   \PassOptionsToPackage{normalsize}{subfigure}
%% before the \usebpackage{subfigure}, to make the fonts the right 
%% size.


%% Declare that figures should be put at end of paper for typesetting
\newcommand{\figuresatend}[1]{%
  \renewcommand{\firstfigref}[1]{\begin{center}--- Insert
      figure~\ref{##1} about here ---\end{center}\ref{##1}}%
  %% Hack to make subfigure package work outside of a float environment
  \makeatletter\newcommand{\@captype}{figure}\makeatother%
}

%% Override this if you need to add your name or manuscript number
%% Accepts one argument, the figure key (suitable for \ref)
\newcommand{\submissionfigdescription}[1]{Figure~\ref{#1}}

% Format a figure for typesetting
% Usage: \insertsubmissionfig[separation]{reference}
\newcommand{\insertsubmissionfig}[2][1in]{%
\newpage\setcounter{subfigure}{0}
\centering
\makebox[0in]{%
\resizebox{7.5in}{!}
{\begin{minipage}{\textwidth}
    \centering%
    \insertfigbody{#2}\\[#1]
    \submissionfigdescription{#2}
  \end{minipage}}}%
}


% Format a caption for typesetting
% Usage: \insertsubmissioncaption{reference}
\newcommand{\insertsubmissioncaption}[1]{%
  \newpage%
  \begin{figure}[h]%
    \renewcommand{\baselinestretch}{1.5}\small\normalsize
    \caption[\insertfigtitle{#1}\figurenote{\protect{\insertfigtype{#1}}}]{%
      \label{#1}%
      \textbf{\insertfigtitle{#1}\figurenote{\protect{\insertfigtype{#1}}}}.
      \raggedright%
      \htmlblock{\insertfigcaption{#1}}}
  \end{figure}%
}



%%%%%%%%%%%%%%%%%%%%%%%%%%%%%%%%%%%%%%%%%%%%%%%%%%%%%%%%%%%%%%%%%%%%%%%%%%%%%%%
%% Tech report title pages
%%
%% Default values; you can (and should) override these using
%% \renewcommand within your own paper.  Often you will need to add
%% different linebreaks to the \techreporttitle and
%% \techreportauthor since they need to fit in such a narrow window.
%%
\newcommand{\techreporttitle}{\title}
\newcommand{\techreportauthor}{\author}
\newcommand{\techreportnumber}{}
\newcommand{\techreportemailaddress}{}
\newcommand{\techreporthttpaddress}{http://www.cs.utexas.edu/users/nn}
\newcommand{\techreportdepartment}{Artificial Intelligence Laboratory}
\newcommand{\universityname}{The University of Texas at Austin}
\newcommand{\universityaddress}{Austin, TX 78712}

%% Call just after your \begin{document} to actually generate the tech
%% report title page.
%%
%% Takes two arguments -- the offset in x and y to correct for the
%% left and top margins, respectively, in the current style.  The
%% values should both be zero for the default 1-inch margins in
%% nnonecolumn.sty; adjust other margins as needed.
%%
\newcommand{\techreporttitlepage}[2]{%
  \newpage
  \thispagestyle{empty}
  \setlength{\unitlength}{1in}%
  \begin{picture}(0,0)(#1,#2)
    \put(1,-2.3){\parbox{4in}{%
      \begin{center}%
        {\textbf{\Large \techreporttitle}}\\[8mm]
        {\textbf{\large \techreportauthor}}\\[8mm]
        \textbf{\large \techreportnumber}
      \end{center}}}
    \put(1,-7.8){\parbox{4in}{%
      \begin{center}%
        {\tt \techreportemailaddress}\\
        {\tt \techreporthttpaddress}\\[8mm]
        \techreportdepartment\\
        \universityname\\
        \universityaddress
      \end{center}}}
  \end{picture}
  \newpage
}


%%%%%%%%%%%%%%%%%%%%%%%%%%%%%%%%%%%%%%%%%%%%%%%%%%%%%%%%%%%%%%%%%%%%%%%%%%%%%%%
% For emacs:
%Local Variables:
%TeX-command-default:"Check"
%End:
