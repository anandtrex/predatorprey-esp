\section{Approach}
In this paper, we consider a toroidal grid world with one predator and one prey. Capture is defined as the predator occupying the same cell as the prey. The prey moves slightly slower than the predator (0.8 times the predator speed). We introduce additional agents into the environment called “hunters”. If the predators come in contact with the hunters, they receive a large negative reward, and hence the predator agents have to learn to stay away from the hunters apart from chasing and capturing the prey. We also fix the behaviour of both the prey and the hunter. The prey always moves away from the predators, while the hunter either moves randomly or moves towards the predator with a fixed probability.

The task decomposition paper discussed above in section []  was the primary
motivation for the work described in this paper. In the predator-prey domain
with hunters, we decompose the task of the predator agent into two parts --
capturing the prey, and avoiding the hunter. We train the predator agents
separately on each of these subtasks i.e. we first train the agent in an environment with only prey and no hunters where it learns to capture prey successfully. Then we train the agent in an environment with only hunters and no prey, where it learns to avoid the hunters successfully. Then we combine the two sub-networks using another network, and train the combiner network on the overall task in an environment where both the prey and the hunters are present. 

The sub-networks and the combiner networks are evolved using both NEAT and multi agent ESP as mentioned above.

The network structure is shown below:
